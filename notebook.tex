
% Default to the notebook output style

    


% Inherit from the specified cell style.




    
\documentclass[11pt]{article}

    
    
    \usepackage[T1]{fontenc}
    % Nicer default font (+ math font) than Computer Modern for most use cases
    \usepackage{mathpazo}

    % Basic figure setup, for now with no caption control since it's done
    % automatically by Pandoc (which extracts ![](path) syntax from Markdown).
    \usepackage{graphicx}
    % We will generate all images so they have a width \maxwidth. This means
    % that they will get their normal width if they fit onto the page, but
    % are scaled down if they would overflow the margins.
    \makeatletter
    \def\maxwidth{\ifdim\Gin@nat@width>\linewidth\linewidth
    \else\Gin@nat@width\fi}
    \makeatother
    \let\Oldincludegraphics\includegraphics
    % Set max figure width to be 80% of text width, for now hardcoded.
    \renewcommand{\includegraphics}[1]{\Oldincludegraphics[width=.8\maxwidth]{#1}}
    % Ensure that by default, figures have no caption (until we provide a
    % proper Figure object with a Caption API and a way to capture that
    % in the conversion process - todo).
    \usepackage{caption}
    \DeclareCaptionLabelFormat{nolabel}{}
    \captionsetup{labelformat=nolabel}

    \usepackage{adjustbox} % Used to constrain images to a maximum size 
    \usepackage{xcolor} % Allow colors to be defined
    \usepackage{enumerate} % Needed for markdown enumerations to work
    \usepackage{geometry} % Used to adjust the document margins
    \usepackage{amsmath} % Equations
    \usepackage{amssymb} % Equations
    \usepackage{textcomp} % defines textquotesingle
    % Hack from http://tex.stackexchange.com/a/47451/13684:
    \AtBeginDocument{%
        \def\PYZsq{\textquotesingle}% Upright quotes in Pygmentized code
    }
    \usepackage{upquote} % Upright quotes for verbatim code
    \usepackage{eurosym} % defines \euro
    \usepackage[mathletters]{ucs} % Extended unicode (utf-8) support
    \usepackage[utf8x]{inputenc} % Allow utf-8 characters in the tex document
    \usepackage{fancyvrb} % verbatim replacement that allows latex
    \usepackage{grffile} % extends the file name processing of package graphics 
                         % to support a larger range 
    % The hyperref package gives us a pdf with properly built
    % internal navigation ('pdf bookmarks' for the table of contents,
    % internal cross-reference links, web links for URLs, etc.)
    \usepackage{hyperref}
    \usepackage{longtable} % longtable support required by pandoc >1.10
    \usepackage{booktabs}  % table support for pandoc > 1.12.2
    \usepackage[inline]{enumitem} % IRkernel/repr support (it uses the enumerate* environment)
    \usepackage[normalem]{ulem} % ulem is needed to support strikethroughs (\sout)
                                % normalem makes italics be italics, not underlines
    

    
    
    % Colors for the hyperref package
    \definecolor{urlcolor}{rgb}{0,.145,.698}
    \definecolor{linkcolor}{rgb}{.71,0.21,0.01}
    \definecolor{citecolor}{rgb}{.12,.54,.11}

    % ANSI colors
    \definecolor{ansi-black}{HTML}{3E424D}
    \definecolor{ansi-black-intense}{HTML}{282C36}
    \definecolor{ansi-red}{HTML}{E75C58}
    \definecolor{ansi-red-intense}{HTML}{B22B31}
    \definecolor{ansi-green}{HTML}{00A250}
    \definecolor{ansi-green-intense}{HTML}{007427}
    \definecolor{ansi-yellow}{HTML}{DDB62B}
    \definecolor{ansi-yellow-intense}{HTML}{B27D12}
    \definecolor{ansi-blue}{HTML}{208FFB}
    \definecolor{ansi-blue-intense}{HTML}{0065CA}
    \definecolor{ansi-magenta}{HTML}{D160C4}
    \definecolor{ansi-magenta-intense}{HTML}{A03196}
    \definecolor{ansi-cyan}{HTML}{60C6C8}
    \definecolor{ansi-cyan-intense}{HTML}{258F8F}
    \definecolor{ansi-white}{HTML}{C5C1B4}
    \definecolor{ansi-white-intense}{HTML}{A1A6B2}

    % commands and environments needed by pandoc snippets
    % extracted from the output of `pandoc -s`
    \providecommand{\tightlist}{%
      \setlength{\itemsep}{0pt}\setlength{\parskip}{0pt}}
    \DefineVerbatimEnvironment{Highlighting}{Verbatim}{commandchars=\\\{\}}
    % Add ',fontsize=\small' for more characters per line
    \newenvironment{Shaded}{}{}
    \newcommand{\KeywordTok}[1]{\textcolor[rgb]{0.00,0.44,0.13}{\textbf{{#1}}}}
    \newcommand{\DataTypeTok}[1]{\textcolor[rgb]{0.56,0.13,0.00}{{#1}}}
    \newcommand{\DecValTok}[1]{\textcolor[rgb]{0.25,0.63,0.44}{{#1}}}
    \newcommand{\BaseNTok}[1]{\textcolor[rgb]{0.25,0.63,0.44}{{#1}}}
    \newcommand{\FloatTok}[1]{\textcolor[rgb]{0.25,0.63,0.44}{{#1}}}
    \newcommand{\CharTok}[1]{\textcolor[rgb]{0.25,0.44,0.63}{{#1}}}
    \newcommand{\StringTok}[1]{\textcolor[rgb]{0.25,0.44,0.63}{{#1}}}
    \newcommand{\CommentTok}[1]{\textcolor[rgb]{0.38,0.63,0.69}{\textit{{#1}}}}
    \newcommand{\OtherTok}[1]{\textcolor[rgb]{0.00,0.44,0.13}{{#1}}}
    \newcommand{\AlertTok}[1]{\textcolor[rgb]{1.00,0.00,0.00}{\textbf{{#1}}}}
    \newcommand{\FunctionTok}[1]{\textcolor[rgb]{0.02,0.16,0.49}{{#1}}}
    \newcommand{\RegionMarkerTok}[1]{{#1}}
    \newcommand{\ErrorTok}[1]{\textcolor[rgb]{1.00,0.00,0.00}{\textbf{{#1}}}}
    \newcommand{\NormalTok}[1]{{#1}}
    
    % Additional commands for more recent versions of Pandoc
    \newcommand{\ConstantTok}[1]{\textcolor[rgb]{0.53,0.00,0.00}{{#1}}}
    \newcommand{\SpecialCharTok}[1]{\textcolor[rgb]{0.25,0.44,0.63}{{#1}}}
    \newcommand{\VerbatimStringTok}[1]{\textcolor[rgb]{0.25,0.44,0.63}{{#1}}}
    \newcommand{\SpecialStringTok}[1]{\textcolor[rgb]{0.73,0.40,0.53}{{#1}}}
    \newcommand{\ImportTok}[1]{{#1}}
    \newcommand{\DocumentationTok}[1]{\textcolor[rgb]{0.73,0.13,0.13}{\textit{{#1}}}}
    \newcommand{\AnnotationTok}[1]{\textcolor[rgb]{0.38,0.63,0.69}{\textbf{\textit{{#1}}}}}
    \newcommand{\CommentVarTok}[1]{\textcolor[rgb]{0.38,0.63,0.69}{\textbf{\textit{{#1}}}}}
    \newcommand{\VariableTok}[1]{\textcolor[rgb]{0.10,0.09,0.49}{{#1}}}
    \newcommand{\ControlFlowTok}[1]{\textcolor[rgb]{0.00,0.44,0.13}{\textbf{{#1}}}}
    \newcommand{\OperatorTok}[1]{\textcolor[rgb]{0.40,0.40,0.40}{{#1}}}
    \newcommand{\BuiltInTok}[1]{{#1}}
    \newcommand{\ExtensionTok}[1]{{#1}}
    \newcommand{\PreprocessorTok}[1]{\textcolor[rgb]{0.74,0.48,0.00}{{#1}}}
    \newcommand{\AttributeTok}[1]{\textcolor[rgb]{0.49,0.56,0.16}{{#1}}}
    \newcommand{\InformationTok}[1]{\textcolor[rgb]{0.38,0.63,0.69}{\textbf{\textit{{#1}}}}}
    \newcommand{\WarningTok}[1]{\textcolor[rgb]{0.38,0.63,0.69}{\textbf{\textit{{#1}}}}}
    
    
    % Define a nice break command that doesn't care if a line doesn't already
    % exist.
    \def\br{\hspace*{\fill} \\* }
    % Math Jax compatability definitions
    \def\gt{>}
    \def\lt{<}
    % Document parameters
    \title{vui\_notebook}
    
    
    

    % Pygments definitions
    
\makeatletter
\def\PY@reset{\let\PY@it=\relax \let\PY@bf=\relax%
    \let\PY@ul=\relax \let\PY@tc=\relax%
    \let\PY@bc=\relax \let\PY@ff=\relax}
\def\PY@tok#1{\csname PY@tok@#1\endcsname}
\def\PY@toks#1+{\ifx\relax#1\empty\else%
    \PY@tok{#1}\expandafter\PY@toks\fi}
\def\PY@do#1{\PY@bc{\PY@tc{\PY@ul{%
    \PY@it{\PY@bf{\PY@ff{#1}}}}}}}
\def\PY#1#2{\PY@reset\PY@toks#1+\relax+\PY@do{#2}}

\expandafter\def\csname PY@tok@w\endcsname{\def\PY@tc##1{\textcolor[rgb]{0.73,0.73,0.73}{##1}}}
\expandafter\def\csname PY@tok@c\endcsname{\let\PY@it=\textit\def\PY@tc##1{\textcolor[rgb]{0.25,0.50,0.50}{##1}}}
\expandafter\def\csname PY@tok@cp\endcsname{\def\PY@tc##1{\textcolor[rgb]{0.74,0.48,0.00}{##1}}}
\expandafter\def\csname PY@tok@k\endcsname{\let\PY@bf=\textbf\def\PY@tc##1{\textcolor[rgb]{0.00,0.50,0.00}{##1}}}
\expandafter\def\csname PY@tok@kp\endcsname{\def\PY@tc##1{\textcolor[rgb]{0.00,0.50,0.00}{##1}}}
\expandafter\def\csname PY@tok@kt\endcsname{\def\PY@tc##1{\textcolor[rgb]{0.69,0.00,0.25}{##1}}}
\expandafter\def\csname PY@tok@o\endcsname{\def\PY@tc##1{\textcolor[rgb]{0.40,0.40,0.40}{##1}}}
\expandafter\def\csname PY@tok@ow\endcsname{\let\PY@bf=\textbf\def\PY@tc##1{\textcolor[rgb]{0.67,0.13,1.00}{##1}}}
\expandafter\def\csname PY@tok@nb\endcsname{\def\PY@tc##1{\textcolor[rgb]{0.00,0.50,0.00}{##1}}}
\expandafter\def\csname PY@tok@nf\endcsname{\def\PY@tc##1{\textcolor[rgb]{0.00,0.00,1.00}{##1}}}
\expandafter\def\csname PY@tok@nc\endcsname{\let\PY@bf=\textbf\def\PY@tc##1{\textcolor[rgb]{0.00,0.00,1.00}{##1}}}
\expandafter\def\csname PY@tok@nn\endcsname{\let\PY@bf=\textbf\def\PY@tc##1{\textcolor[rgb]{0.00,0.00,1.00}{##1}}}
\expandafter\def\csname PY@tok@ne\endcsname{\let\PY@bf=\textbf\def\PY@tc##1{\textcolor[rgb]{0.82,0.25,0.23}{##1}}}
\expandafter\def\csname PY@tok@nv\endcsname{\def\PY@tc##1{\textcolor[rgb]{0.10,0.09,0.49}{##1}}}
\expandafter\def\csname PY@tok@no\endcsname{\def\PY@tc##1{\textcolor[rgb]{0.53,0.00,0.00}{##1}}}
\expandafter\def\csname PY@tok@nl\endcsname{\def\PY@tc##1{\textcolor[rgb]{0.63,0.63,0.00}{##1}}}
\expandafter\def\csname PY@tok@ni\endcsname{\let\PY@bf=\textbf\def\PY@tc##1{\textcolor[rgb]{0.60,0.60,0.60}{##1}}}
\expandafter\def\csname PY@tok@na\endcsname{\def\PY@tc##1{\textcolor[rgb]{0.49,0.56,0.16}{##1}}}
\expandafter\def\csname PY@tok@nt\endcsname{\let\PY@bf=\textbf\def\PY@tc##1{\textcolor[rgb]{0.00,0.50,0.00}{##1}}}
\expandafter\def\csname PY@tok@nd\endcsname{\def\PY@tc##1{\textcolor[rgb]{0.67,0.13,1.00}{##1}}}
\expandafter\def\csname PY@tok@s\endcsname{\def\PY@tc##1{\textcolor[rgb]{0.73,0.13,0.13}{##1}}}
\expandafter\def\csname PY@tok@sd\endcsname{\let\PY@it=\textit\def\PY@tc##1{\textcolor[rgb]{0.73,0.13,0.13}{##1}}}
\expandafter\def\csname PY@tok@si\endcsname{\let\PY@bf=\textbf\def\PY@tc##1{\textcolor[rgb]{0.73,0.40,0.53}{##1}}}
\expandafter\def\csname PY@tok@se\endcsname{\let\PY@bf=\textbf\def\PY@tc##1{\textcolor[rgb]{0.73,0.40,0.13}{##1}}}
\expandafter\def\csname PY@tok@sr\endcsname{\def\PY@tc##1{\textcolor[rgb]{0.73,0.40,0.53}{##1}}}
\expandafter\def\csname PY@tok@ss\endcsname{\def\PY@tc##1{\textcolor[rgb]{0.10,0.09,0.49}{##1}}}
\expandafter\def\csname PY@tok@sx\endcsname{\def\PY@tc##1{\textcolor[rgb]{0.00,0.50,0.00}{##1}}}
\expandafter\def\csname PY@tok@m\endcsname{\def\PY@tc##1{\textcolor[rgb]{0.40,0.40,0.40}{##1}}}
\expandafter\def\csname PY@tok@gh\endcsname{\let\PY@bf=\textbf\def\PY@tc##1{\textcolor[rgb]{0.00,0.00,0.50}{##1}}}
\expandafter\def\csname PY@tok@gu\endcsname{\let\PY@bf=\textbf\def\PY@tc##1{\textcolor[rgb]{0.50,0.00,0.50}{##1}}}
\expandafter\def\csname PY@tok@gd\endcsname{\def\PY@tc##1{\textcolor[rgb]{0.63,0.00,0.00}{##1}}}
\expandafter\def\csname PY@tok@gi\endcsname{\def\PY@tc##1{\textcolor[rgb]{0.00,0.63,0.00}{##1}}}
\expandafter\def\csname PY@tok@gr\endcsname{\def\PY@tc##1{\textcolor[rgb]{1.00,0.00,0.00}{##1}}}
\expandafter\def\csname PY@tok@ge\endcsname{\let\PY@it=\textit}
\expandafter\def\csname PY@tok@gs\endcsname{\let\PY@bf=\textbf}
\expandafter\def\csname PY@tok@gp\endcsname{\let\PY@bf=\textbf\def\PY@tc##1{\textcolor[rgb]{0.00,0.00,0.50}{##1}}}
\expandafter\def\csname PY@tok@go\endcsname{\def\PY@tc##1{\textcolor[rgb]{0.53,0.53,0.53}{##1}}}
\expandafter\def\csname PY@tok@gt\endcsname{\def\PY@tc##1{\textcolor[rgb]{0.00,0.27,0.87}{##1}}}
\expandafter\def\csname PY@tok@err\endcsname{\def\PY@bc##1{\setlength{\fboxsep}{0pt}\fcolorbox[rgb]{1.00,0.00,0.00}{1,1,1}{\strut ##1}}}
\expandafter\def\csname PY@tok@kc\endcsname{\let\PY@bf=\textbf\def\PY@tc##1{\textcolor[rgb]{0.00,0.50,0.00}{##1}}}
\expandafter\def\csname PY@tok@kd\endcsname{\let\PY@bf=\textbf\def\PY@tc##1{\textcolor[rgb]{0.00,0.50,0.00}{##1}}}
\expandafter\def\csname PY@tok@kn\endcsname{\let\PY@bf=\textbf\def\PY@tc##1{\textcolor[rgb]{0.00,0.50,0.00}{##1}}}
\expandafter\def\csname PY@tok@kr\endcsname{\let\PY@bf=\textbf\def\PY@tc##1{\textcolor[rgb]{0.00,0.50,0.00}{##1}}}
\expandafter\def\csname PY@tok@bp\endcsname{\def\PY@tc##1{\textcolor[rgb]{0.00,0.50,0.00}{##1}}}
\expandafter\def\csname PY@tok@fm\endcsname{\def\PY@tc##1{\textcolor[rgb]{0.00,0.00,1.00}{##1}}}
\expandafter\def\csname PY@tok@vc\endcsname{\def\PY@tc##1{\textcolor[rgb]{0.10,0.09,0.49}{##1}}}
\expandafter\def\csname PY@tok@vg\endcsname{\def\PY@tc##1{\textcolor[rgb]{0.10,0.09,0.49}{##1}}}
\expandafter\def\csname PY@tok@vi\endcsname{\def\PY@tc##1{\textcolor[rgb]{0.10,0.09,0.49}{##1}}}
\expandafter\def\csname PY@tok@vm\endcsname{\def\PY@tc##1{\textcolor[rgb]{0.10,0.09,0.49}{##1}}}
\expandafter\def\csname PY@tok@sa\endcsname{\def\PY@tc##1{\textcolor[rgb]{0.73,0.13,0.13}{##1}}}
\expandafter\def\csname PY@tok@sb\endcsname{\def\PY@tc##1{\textcolor[rgb]{0.73,0.13,0.13}{##1}}}
\expandafter\def\csname PY@tok@sc\endcsname{\def\PY@tc##1{\textcolor[rgb]{0.73,0.13,0.13}{##1}}}
\expandafter\def\csname PY@tok@dl\endcsname{\def\PY@tc##1{\textcolor[rgb]{0.73,0.13,0.13}{##1}}}
\expandafter\def\csname PY@tok@s2\endcsname{\def\PY@tc##1{\textcolor[rgb]{0.73,0.13,0.13}{##1}}}
\expandafter\def\csname PY@tok@sh\endcsname{\def\PY@tc##1{\textcolor[rgb]{0.73,0.13,0.13}{##1}}}
\expandafter\def\csname PY@tok@s1\endcsname{\def\PY@tc##1{\textcolor[rgb]{0.73,0.13,0.13}{##1}}}
\expandafter\def\csname PY@tok@mb\endcsname{\def\PY@tc##1{\textcolor[rgb]{0.40,0.40,0.40}{##1}}}
\expandafter\def\csname PY@tok@mf\endcsname{\def\PY@tc##1{\textcolor[rgb]{0.40,0.40,0.40}{##1}}}
\expandafter\def\csname PY@tok@mh\endcsname{\def\PY@tc##1{\textcolor[rgb]{0.40,0.40,0.40}{##1}}}
\expandafter\def\csname PY@tok@mi\endcsname{\def\PY@tc##1{\textcolor[rgb]{0.40,0.40,0.40}{##1}}}
\expandafter\def\csname PY@tok@il\endcsname{\def\PY@tc##1{\textcolor[rgb]{0.40,0.40,0.40}{##1}}}
\expandafter\def\csname PY@tok@mo\endcsname{\def\PY@tc##1{\textcolor[rgb]{0.40,0.40,0.40}{##1}}}
\expandafter\def\csname PY@tok@ch\endcsname{\let\PY@it=\textit\def\PY@tc##1{\textcolor[rgb]{0.25,0.50,0.50}{##1}}}
\expandafter\def\csname PY@tok@cm\endcsname{\let\PY@it=\textit\def\PY@tc##1{\textcolor[rgb]{0.25,0.50,0.50}{##1}}}
\expandafter\def\csname PY@tok@cpf\endcsname{\let\PY@it=\textit\def\PY@tc##1{\textcolor[rgb]{0.25,0.50,0.50}{##1}}}
\expandafter\def\csname PY@tok@c1\endcsname{\let\PY@it=\textit\def\PY@tc##1{\textcolor[rgb]{0.25,0.50,0.50}{##1}}}
\expandafter\def\csname PY@tok@cs\endcsname{\let\PY@it=\textit\def\PY@tc##1{\textcolor[rgb]{0.25,0.50,0.50}{##1}}}

\def\PYZbs{\char`\\}
\def\PYZus{\char`\_}
\def\PYZob{\char`\{}
\def\PYZcb{\char`\}}
\def\PYZca{\char`\^}
\def\PYZam{\char`\&}
\def\PYZlt{\char`\<}
\def\PYZgt{\char`\>}
\def\PYZsh{\char`\#}
\def\PYZpc{\char`\%}
\def\PYZdl{\char`\$}
\def\PYZhy{\char`\-}
\def\PYZsq{\char`\'}
\def\PYZdq{\char`\"}
\def\PYZti{\char`\~}
% for compatibility with earlier versions
\def\PYZat{@}
\def\PYZlb{[}
\def\PYZrb{]}
\makeatother


    % Exact colors from NB
    \definecolor{incolor}{rgb}{0.0, 0.0, 0.5}
    \definecolor{outcolor}{rgb}{0.545, 0.0, 0.0}



    
    % Prevent overflowing lines due to hard-to-break entities
    \sloppy 
    % Setup hyperref package
    \hypersetup{
      breaklinks=true,  % so long urls are correctly broken across lines
      colorlinks=true,
      urlcolor=urlcolor,
      linkcolor=linkcolor,
      citecolor=citecolor,
      }
    % Slightly bigger margins than the latex defaults
    
    \geometry{verbose,tmargin=1in,bmargin=1in,lmargin=1in,rmargin=1in}
    
    

    \begin{document}
    
    
    \maketitle
    
    

    
    \section{Artificial Intelligence
Nanodegree}\label{artificial-intelligence-nanodegree}

\subsection{Voice User Interfaces}\label{voice-user-interfaces}

\subsection{Project: Speech Recognition with Neural
Networks}\label{project-speech-recognition-with-neural-networks}

\begin{center}\rule{0.5\linewidth}{\linethickness}\end{center}

In this notebook, some template code has already been provided for you,
and you will need to implement additional functionality to successfully
complete this project. You will not need to modify the included code
beyond what is requested. Sections that begin with
\textbf{'(IMPLEMENTATION)'} in the header indicate that the following
blocks of code will require additional functionality which you must
provide. Please be sure to read the instructions carefully!

\begin{quote}
\textbf{Note}: Once you have completed all of the code implementations,
you need to finalize your work by exporting the Jupyter Notebook as an
HTML document. Before exporting the notebook to html, all of the code
cells need to have been run so that reviewers can see the final
implementation and output. You can then export the notebook by using the
menu above and navigating to \n", "\textbf{File -\textgreater{} Download
as -\textgreater{} HTML (.html)}. Include the finished document along
with this notebook as your submission.
\end{quote}

In addition to implementing code, there will be questions that you must
answer which relate to the project and your implementation. Each section
where you will answer a question is preceded by a \textbf{'Question X'}
header. Carefully read each question and provide thorough answers in the
following text boxes that begin with \textbf{'Answer:'}. Your project
submission will be evaluated based on your answers to each of the
questions and the implementation you provide.

\begin{quote}
\textbf{Note:} Code and Markdown cells can be executed using the
\textbf{Shift + Enter} keyboard shortcut. Markdown cells can be edited
by double-clicking the cell to enter edit mode.
\end{quote}

The rubric contains \emph{optional} "Stand Out Suggestions" for
enhancing the project beyond the minimum requirements. If you decide to
pursue the "Stand Out Suggestions", you should include the code in this
Jupyter notebook.

\begin{center}\rule{0.5\linewidth}{\linethickness}\end{center}

\subsection{Introduction}\label{introduction}

In this notebook, you will build a deep neural network that functions as
part of an end-to-end automatic speech recognition (ASR) pipeline! Your
completed pipeline will accept raw audio as input and return a predicted
transcription of the spoken language. The full pipeline is summarized in
the figure below.

\begin{itemize}
\tightlist
\item
  \textbf{STEP 1} is a pre-processing step that converts raw audio to
  one of two feature representations that are commonly used for ASR.
\item
  \textbf{STEP 2} is an acoustic model which accepts audio features as
  input and returns a probability distribution over all potential
  transcriptions. After learning about the basic types of neural
  networks that are often used for acoustic modeling, you will engage in
  your own investigations, to design your own acoustic model!
\item
  \textbf{STEP 3} in the pipeline takes the output from the acoustic
  model and returns a predicted transcription.
\end{itemize}

Feel free to use the links below to navigate the notebook: -
Section \ref{thedata} - Section \ref{step1}: Acoustic Features for
Speech Recognition - Section \ref{step2}: Deep Neural Networks for
Acoustic Modeling - Section \ref{model0}: RNN - Section \ref{model1}:
RNN + TimeDistributed Dense - Section \ref{model2}: CNN + RNN +
TimeDistributed Dense - Section \ref{model3}: Deeper RNN +
TimeDistributed Dense - Section \ref{model4}: Bidirectional RNN +
TimeDistributed Dense - Section \ref{model5} - Section \ref{compare} -
Section \ref{final} - Section \ref{step3}: Obtain Predictions

 \#\# The Data

We begin by investigating the dataset that will be used to train and
evaluate your pipeline.
\href{http://www.danielpovey.com/files/2015_icassp_librispeech.pdf}{LibriSpeech}
is a large corpus of English-read speech, designed for training and
evaluating models for ASR. The dataset contains 1000 hours of speech
derived from audiobooks. We will work with a small subset in this
project, since larger-scale data would take a long while to train.
However, after completing this project, if you are interested in
exploring further, you are encouraged to work with more of the data that
is provided \href{http://www.openslr.org/12/}{online}.

In the code cells below, you will use the \texttt{vis\_train\_features}
module to visualize a training example. The supplied argument
\texttt{index=0} tells the module to extract the first example in the
training set. (You are welcome to change \texttt{index=0} to point to a
different training example, if you like, but please \textbf{DO NOT}
amend any other code in the cell.) The returned variables are: -
\texttt{vis\_text} - transcribed text (label) for the training example.
- \texttt{vis\_raw\_audio} - raw audio waveform for the training
example. - \texttt{vis\_mfcc\_feature} - mel-frequency cepstral
coefficients (MFCCs) for the training example. -
\texttt{vis\_spectrogram\_feature} - spectrogram for the training
example. - \texttt{vis\_audio\_path} - the file path to the training
example.

    \begin{Verbatim}[commandchars=\\\{\}]
{\color{incolor}In [{\color{incolor}1}]:} \PY{k+kn}{from} \PY{n+nn}{data\PYZus{}generator} \PY{k}{import} \PY{n}{vis\PYZus{}train\PYZus{}features}
        
        \PY{c+c1}{\PYZsh{} extract label and audio features for a single training example}
        \PY{n}{vis\PYZus{}text}\PY{p}{,} \PY{n}{vis\PYZus{}raw\PYZus{}audio}\PY{p}{,} \PY{n}{vis\PYZus{}mfcc\PYZus{}feature}\PY{p}{,} \PY{n}{vis\PYZus{}spectrogram\PYZus{}feature}\PY{p}{,} \PY{n}{vis\PYZus{}audio\PYZus{}path} \PY{o}{=} \PY{n}{vis\PYZus{}train\PYZus{}features}\PY{p}{(}\PY{p}{)}
\end{Verbatim}


    \begin{Verbatim}[commandchars=\\\{\}]
There are 2023 total training examples.

    \end{Verbatim}

    The following code cell visualizes the audio waveform for your chosen
example, along with the corresponding transcript. You also have the
option to play the audio in the notebook!

    \begin{Verbatim}[commandchars=\\\{\}]
{\color{incolor}In [{\color{incolor}2}]:} \PY{k+kn}{from} \PY{n+nn}{IPython}\PY{n+nn}{.}\PY{n+nn}{display} \PY{k}{import} \PY{n}{Markdown}\PY{p}{,} \PY{n}{display}
        \PY{k+kn}{from} \PY{n+nn}{data\PYZus{}generator} \PY{k}{import} \PY{n}{vis\PYZus{}train\PYZus{}features}\PY{p}{,} \PY{n}{plot\PYZus{}raw\PYZus{}audio}
        \PY{k+kn}{from} \PY{n+nn}{IPython}\PY{n+nn}{.}\PY{n+nn}{display} \PY{k}{import} \PY{n}{Audio}
        \PY{o}{\PYZpc{}}\PY{k}{matplotlib} inline
        
        \PY{c+c1}{\PYZsh{} plot audio signal}
        \PY{n}{plot\PYZus{}raw\PYZus{}audio}\PY{p}{(}\PY{n}{vis\PYZus{}raw\PYZus{}audio}\PY{p}{)}
        \PY{c+c1}{\PYZsh{} print length of audio signal}
        \PY{n}{display}\PY{p}{(}\PY{n}{Markdown}\PY{p}{(}\PY{l+s+s1}{\PYZsq{}}\PY{l+s+s1}{**Shape of Audio Signal** : }\PY{l+s+s1}{\PYZsq{}} \PY{o}{+} \PY{n+nb}{str}\PY{p}{(}\PY{n}{vis\PYZus{}raw\PYZus{}audio}\PY{o}{.}\PY{n}{shape}\PY{p}{)}\PY{p}{)}\PY{p}{)}
        \PY{c+c1}{\PYZsh{} print transcript corresponding to audio clip}
        \PY{n}{display}\PY{p}{(}\PY{n}{Markdown}\PY{p}{(}\PY{l+s+s1}{\PYZsq{}}\PY{l+s+s1}{**Transcript** : }\PY{l+s+s1}{\PYZsq{}} \PY{o}{+} \PY{n+nb}{str}\PY{p}{(}\PY{n}{vis\PYZus{}text}\PY{p}{)}\PY{p}{)}\PY{p}{)}
        \PY{c+c1}{\PYZsh{} play the audio file}
        \PY{n}{Audio}\PY{p}{(}\PY{n}{vis\PYZus{}audio\PYZus{}path}\PY{p}{)}
\end{Verbatim}


    \begin{center}
    \adjustimage{max size={0.9\linewidth}{0.9\paperheight}}{output_3_0.png}
    \end{center}
    { \hspace*{\fill} \\}
    
    \textbf{Shape of Audio Signal} : (129103,)

    
    \textbf{Transcript} : mister quilter is the apostle of the middle
classes and we are glad to welcome his gospel

    
\begin{Verbatim}[commandchars=\\\{\}]
{\color{outcolor}Out[{\color{outcolor}2}]:} <IPython.lib.display.Audio object>
\end{Verbatim}
            
     \#\# STEP 1: Acoustic Features for Speech Recognition

For this project, you won't use the raw audio waveform as input to your
model. Instead, we provide code that first performs a pre-processing
step to convert the raw audio to a feature representation that has
historically proven successful for ASR models. Your acoustic model will
accept the feature representation as input.

In this project, you will explore two possible feature representations.
\emph{After completing the project}, if you'd like to read more about
deep learning architectures that can accept raw audio input, you are
encouraged to explore this
\href{https://pdfs.semanticscholar.org/a566/cd4a8623d661a4931814d9dffc72ecbf63c4.pdf}{research
paper}.

\subsubsection{Spectrograms}\label{spectrograms}

The first option for an audio feature representation is the
\href{https://www.youtube.com/watch?v=_FatxGN3vAM}{spectrogram}. In
order to complete this project, you will \textbf{not} need to dig deeply
into the details of how a spectrogram is calculated; but, if you are
curious, the code for calculating the spectrogram was borrowed from
\href{https://github.com/baidu-research/ba-dls-deepspeech}{this
repository}. The implementation appears in the \texttt{utils.py} file in
your repository.

The code that we give you returns the spectrogram as a 2D tensor, where
the first (\emph{vertical}) dimension indexes time, and the second
(\emph{horizontal}) dimension indexes frequency. To speed the
convergence of your algorithm, we have also normalized the spectrogram.
(You can see this quickly in the visualization below by noting that the
mean value hovers around zero, and most entries in the tensor assume
values close to zero.)

    \begin{Verbatim}[commandchars=\\\{\}]
{\color{incolor}In [{\color{incolor}3}]:} \PY{k+kn}{from} \PY{n+nn}{data\PYZus{}generator} \PY{k}{import} \PY{n}{plot\PYZus{}spectrogram\PYZus{}feature}
        
        \PY{c+c1}{\PYZsh{} plot normalized spectrogram}
        \PY{n}{plot\PYZus{}spectrogram\PYZus{}feature}\PY{p}{(}\PY{n}{vis\PYZus{}spectrogram\PYZus{}feature}\PY{p}{)}
        \PY{c+c1}{\PYZsh{} print shape of spectrogram}
        \PY{n}{display}\PY{p}{(}\PY{n}{Markdown}\PY{p}{(}\PY{l+s+s1}{\PYZsq{}}\PY{l+s+s1}{**Shape of Spectrogram** : }\PY{l+s+s1}{\PYZsq{}} \PY{o}{+} \PY{n+nb}{str}\PY{p}{(}\PY{n}{vis\PYZus{}spectrogram\PYZus{}feature}\PY{o}{.}\PY{n}{shape}\PY{p}{)}\PY{p}{)}\PY{p}{)}
\end{Verbatim}


    \begin{center}
    \adjustimage{max size={0.9\linewidth}{0.9\paperheight}}{output_5_0.png}
    \end{center}
    { \hspace*{\fill} \\}
    
    \textbf{Shape of Spectrogram} : (584, 161)

    
    \subsubsection{Mel-Frequency Cepstral Coefficients
(MFCCs)}\label{mel-frequency-cepstral-coefficients-mfccs}

The second option for an audio feature representation is
\href{https://en.wikipedia.org/wiki/Mel-frequency_cepstrum}{MFCCs}. You
do \textbf{not} need to dig deeply into the details of how MFCCs are
calculated, but if you would like more information, you are welcome to
peruse the
\href{https://github.com/jameslyons/python_speech_features}{documentation}
of the \texttt{python\_speech\_features} Python package. Just as with
the spectrogram features, the MFCCs are normalized in the supplied code.

The main idea behind MFCC features is the same as spectrogram features:
at each time window, the MFCC feature yields a feature vector that
characterizes the sound within the window. Note that the MFCC feature is
much lower-dimensional than the spectrogram feature, which could help an
acoustic model to avoid overfitting to the training dataset.

    \begin{Verbatim}[commandchars=\\\{\}]
{\color{incolor}In [{\color{incolor}4}]:} \PY{k+kn}{from} \PY{n+nn}{data\PYZus{}generator} \PY{k}{import} \PY{n}{plot\PYZus{}mfcc\PYZus{}feature}
        
        \PY{c+c1}{\PYZsh{} plot normalized MFCC}
        \PY{n}{plot\PYZus{}mfcc\PYZus{}feature}\PY{p}{(}\PY{n}{vis\PYZus{}mfcc\PYZus{}feature}\PY{p}{)}
        \PY{c+c1}{\PYZsh{} print shape of MFCC}
        \PY{n}{display}\PY{p}{(}\PY{n}{Markdown}\PY{p}{(}\PY{l+s+s1}{\PYZsq{}}\PY{l+s+s1}{**Shape of MFCC** : }\PY{l+s+s1}{\PYZsq{}} \PY{o}{+} \PY{n+nb}{str}\PY{p}{(}\PY{n}{vis\PYZus{}mfcc\PYZus{}feature}\PY{o}{.}\PY{n}{shape}\PY{p}{)}\PY{p}{)}\PY{p}{)}
\end{Verbatim}


    \begin{center}
    \adjustimage{max size={0.9\linewidth}{0.9\paperheight}}{output_7_0.png}
    \end{center}
    { \hspace*{\fill} \\}
    
    \textbf{Shape of MFCC} : (584, 13)

    
    When you construct your pipeline, you will be able to choose to use
either spectrogram or MFCC features. If you would like to see different
implementations that make use of MFCCs and/or spectrograms, please check
out the links below: - This
\href{https://github.com/baidu-research/ba-dls-deepspeech}{repository}
uses spectrograms. - This
\href{https://github.com/mozilla/DeepSpeech}{repository} uses MFCCs. -
This
\href{https://github.com/buriburisuri/speech-to-text-wavenet}{repository}
also uses MFCCs. - This
\href{https://github.com/pannous/tensorflow-speech-recognition/blob/master/speech_data.py}{repository}
experiments with raw audio, spectrograms, and MFCCs as features.

     \#\# STEP 2: Deep Neural Networks for Acoustic Modeling

In this section, you will experiment with various neural network
architectures for acoustic modeling.

You will begin by training five relatively simple architectures.
\textbf{Model 0} is provided for you. You will write code to implement
\textbf{Models 1}, \textbf{2}, \textbf{3}, and \textbf{4}. If you would
like to experiment further, you are welcome to create and train more
models under the \textbf{Models 5+} heading.

All models will be specified in the \texttt{sample\_models.py} file.
After importing the \texttt{sample\_models} module, you will train your
architectures in the notebook.

After experimenting with the five simple architectures, you will have
the opportunity to compare their performance. Based on your findings,
you will construct a deeper architecture that is designed to outperform
all of the shallow models.

For your convenience, we have designed the notebook so that each model
can be specified and trained on separate occasions. That is, say you
decide to take a break from the notebook after training \textbf{Model
1}. Then, you need not re-execute all prior code cells in the notebook
before training \textbf{Model 2}. You need only re-execute the code cell
below, that is marked with
\textbf{\texttt{RUN\ THIS\ CODE\ CELL\ IF\ YOU\ ARE\ RESUMING\ THE\ NOTEBOOK\ AFTER\ A\ BREAK}},
before transitioning to the code cells corresponding to \textbf{Model
2}.

    \begin{Verbatim}[commandchars=\\\{\}]
{\color{incolor}In [{\color{incolor}1}]:} \PY{c+c1}{\PYZsh{}\PYZsh{}\PYZsh{}\PYZsh{}\PYZsh{}\PYZsh{}\PYZsh{}\PYZsh{}\PYZsh{}\PYZsh{}\PYZsh{}\PYZsh{}\PYZsh{}\PYZsh{}\PYZsh{}\PYZsh{}\PYZsh{}\PYZsh{}\PYZsh{}\PYZsh{}\PYZsh{}\PYZsh{}\PYZsh{}\PYZsh{}\PYZsh{}\PYZsh{}\PYZsh{}\PYZsh{}\PYZsh{}\PYZsh{}\PYZsh{}\PYZsh{}\PYZsh{}\PYZsh{}\PYZsh{}\PYZsh{}\PYZsh{}\PYZsh{}\PYZsh{}\PYZsh{}\PYZsh{}\PYZsh{}\PYZsh{}\PYZsh{}\PYZsh{}\PYZsh{}\PYZsh{}\PYZsh{}\PYZsh{}\PYZsh{}\PYZsh{}\PYZsh{}\PYZsh{}\PYZsh{}\PYZsh{}\PYZsh{}\PYZsh{}\PYZsh{}\PYZsh{}\PYZsh{}\PYZsh{}\PYZsh{}\PYZsh{}\PYZsh{}\PYZsh{}\PYZsh{}\PYZsh{}\PYZsh{}\PYZsh{}}
        \PY{c+c1}{\PYZsh{} RUN THIS CODE CELL IF YOU ARE RESUMING THE NOTEBOOK AFTER A BREAK \PYZsh{}}
        \PY{c+c1}{\PYZsh{}\PYZsh{}\PYZsh{}\PYZsh{}\PYZsh{}\PYZsh{}\PYZsh{}\PYZsh{}\PYZsh{}\PYZsh{}\PYZsh{}\PYZsh{}\PYZsh{}\PYZsh{}\PYZsh{}\PYZsh{}\PYZsh{}\PYZsh{}\PYZsh{}\PYZsh{}\PYZsh{}\PYZsh{}\PYZsh{}\PYZsh{}\PYZsh{}\PYZsh{}\PYZsh{}\PYZsh{}\PYZsh{}\PYZsh{}\PYZsh{}\PYZsh{}\PYZsh{}\PYZsh{}\PYZsh{}\PYZsh{}\PYZsh{}\PYZsh{}\PYZsh{}\PYZsh{}\PYZsh{}\PYZsh{}\PYZsh{}\PYZsh{}\PYZsh{}\PYZsh{}\PYZsh{}\PYZsh{}\PYZsh{}\PYZsh{}\PYZsh{}\PYZsh{}\PYZsh{}\PYZsh{}\PYZsh{}\PYZsh{}\PYZsh{}\PYZsh{}\PYZsh{}\PYZsh{}\PYZsh{}\PYZsh{}\PYZsh{}\PYZsh{}\PYZsh{}\PYZsh{}\PYZsh{}\PYZsh{}\PYZsh{}}
        
        \PY{c+c1}{\PYZsh{} allocate 50\PYZpc{} of GPU memory (if you like, feel free to change this)}
        \PY{k+kn}{from} \PY{n+nn}{keras}\PY{n+nn}{.}\PY{n+nn}{backend}\PY{n+nn}{.}\PY{n+nn}{tensorflow\PYZus{}backend} \PY{k}{import} \PY{n}{set\PYZus{}session}
        \PY{k+kn}{import} \PY{n+nn}{tensorflow} \PY{k}{as} \PY{n+nn}{tf} 
        \PY{n}{config} \PY{o}{=} \PY{n}{tf}\PY{o}{.}\PY{n}{ConfigProto}\PY{p}{(}\PY{p}{)}
        \PY{n}{config}\PY{o}{.}\PY{n}{gpu\PYZus{}options}\PY{o}{.}\PY{n}{per\PYZus{}process\PYZus{}gpu\PYZus{}memory\PYZus{}fraction} \PY{o}{=} \PY{l+m+mf}{0.5}
        \PY{n}{set\PYZus{}session}\PY{p}{(}\PY{n}{tf}\PY{o}{.}\PY{n}{Session}\PY{p}{(}\PY{n}{config}\PY{o}{=}\PY{n}{config}\PY{p}{)}\PY{p}{)}
        
        \PY{c+c1}{\PYZsh{} watch for any changes in the sample\PYZus{}models module, and reload it automatically}
        \PY{o}{\PYZpc{}}\PY{k}{load\PYZus{}ext} autoreload
        \PY{o}{\PYZpc{}}\PY{k}{autoreload} 2
        \PY{c+c1}{\PYZsh{} import NN architectures for speech recognition}
        \PY{k+kn}{from} \PY{n+nn}{sample\PYZus{}models} \PY{k}{import} \PY{o}{*}
        \PY{c+c1}{\PYZsh{} import function for training acoustic model}
        \PY{k+kn}{from} \PY{n+nn}{train\PYZus{}utils} \PY{k}{import} \PY{n}{train\PYZus{}model}
\end{Verbatim}


    \begin{Verbatim}[commandchars=\\\{\}]
Using TensorFlow backend.

    \end{Verbatim}

     \#\#\# Model 0: RNN

Given their effectiveness in modeling sequential data, the first
acoustic model you will use is an RNN. As shown in the figure below, the
RNN we supply to you will take the time sequence of audio features as
input.

At each time step, the speaker pronounces one of 28 possible characters,
including each of the 26 letters in the English alphabet, along with a
space character (" "), and an apostrophe (').

The output of the RNN at each time step is a vector of probabilities
with 29 entries, where the \(i\)-th entry encodes the probability that
the \(i\)-th character is spoken in the time sequence. (The extra 29th
character is an empty "character" used to pad training examples within
batches containing uneven lengths.) If you would like to peek under the
hood at how characters are mapped to indices in the probability vector,
look at the \texttt{char\_map.py} file in the repository. The figure
below shows an equivalent, rolled depiction of the RNN that shows the
output layer in greater detail.

The model has already been specified for you in Keras. To import it, you
need only run the code cell below.

    \begin{Verbatim}[commandchars=\\\{\}]
{\color{incolor}In [{\color{incolor}6}]:} \PY{n}{model\PYZus{}0} \PY{o}{=} \PY{n}{simple\PYZus{}rnn\PYZus{}model}\PY{p}{(}\PY{n}{input\PYZus{}dim}\PY{o}{=}\PY{l+m+mi}{161}\PY{p}{)} \PY{c+c1}{\PYZsh{} change to 13 if you would like to use MFCC features}
\end{Verbatim}


    \begin{Verbatim}[commandchars=\\\{\}]
\_\_\_\_\_\_\_\_\_\_\_\_\_\_\_\_\_\_\_\_\_\_\_\_\_\_\_\_\_\_\_\_\_\_\_\_\_\_\_\_\_\_\_\_\_\_\_\_\_\_\_\_\_\_\_\_\_\_\_\_\_\_\_\_\_
Layer (type)                 Output Shape              Param \#   
=================================================================
the\_input (InputLayer)       (None, None, 161)         0         
\_\_\_\_\_\_\_\_\_\_\_\_\_\_\_\_\_\_\_\_\_\_\_\_\_\_\_\_\_\_\_\_\_\_\_\_\_\_\_\_\_\_\_\_\_\_\_\_\_\_\_\_\_\_\_\_\_\_\_\_\_\_\_\_\_
rnn (GRU)                    (None, None, 29)          16617     
\_\_\_\_\_\_\_\_\_\_\_\_\_\_\_\_\_\_\_\_\_\_\_\_\_\_\_\_\_\_\_\_\_\_\_\_\_\_\_\_\_\_\_\_\_\_\_\_\_\_\_\_\_\_\_\_\_\_\_\_\_\_\_\_\_
softmax (Activation)         (None, None, 29)          0         
=================================================================
Total params: 16,617
Trainable params: 16,617
Non-trainable params: 0
\_\_\_\_\_\_\_\_\_\_\_\_\_\_\_\_\_\_\_\_\_\_\_\_\_\_\_\_\_\_\_\_\_\_\_\_\_\_\_\_\_\_\_\_\_\_\_\_\_\_\_\_\_\_\_\_\_\_\_\_\_\_\_\_\_
None

    \end{Verbatim}

    As explored in the lesson, you will train the acoustic model with the
\href{http://www.cs.toronto.edu/~graves/icml_2006.pdf}{CTC loss}
criterion. Custom loss functions take a bit of hacking in Keras, and so
we have implemented the CTC loss function for you, so that you can focus
on trying out as many deep learning architectures as possible :). If
you'd like to peek at the implementation details, look at the
\texttt{add\_ctc\_loss} function within the \texttt{train\_utils.py}
file in the repository.

To train your architecture, you will use the \texttt{train\_model}
function within the \texttt{train\_utils} module; it has already been
imported in one of the above code cells. The \texttt{train\_model}
function takes three \textbf{required} arguments: -
\texttt{input\_to\_softmax} - a Keras model instance. -
\texttt{pickle\_path} - the name of the pickle file where the loss
history will be saved. - \texttt{save\_model\_path} - the name of the
HDF5 file where the model will be saved.

If we have already supplied values for \texttt{input\_to\_softmax},
\texttt{pickle\_path}, and \texttt{save\_model\_path}, please \textbf{DO
NOT} modify these values.

There are several \textbf{optional} arguments that allow you to have
more control over the training process. You are welcome to, but not
required to, supply your own values for these arguments. -
\texttt{minibatch\_size} - the size of the minibatches that are
generated while training the model (default: \texttt{20}). -
\texttt{spectrogram} - Boolean value dictating whether spectrogram
(\texttt{True}) or MFCC (\texttt{False}) features are used for training
(default: \texttt{True}). - \texttt{mfcc\_dim} - the size of the feature
dimension to use when generating MFCC features (default: \texttt{13}). -
\texttt{optimizer} - the Keras optimizer used to train the model
(default:
\texttt{SGD(lr=0.02,\ decay=1e-6,\ momentum=0.9,\ nesterov=True,\ clipnorm=5)}).\\
- \texttt{epochs} - the number of epochs to use to train the model
(default: \texttt{20}). If you choose to modify this parameter, make
sure that it is \emph{at least} 20. - \texttt{verbose} - controls the
verbosity of the training output in the \texttt{model.fit\_generator}
method (default: \texttt{1}). - \texttt{sort\_by\_duration} - Boolean
value dictating whether the training and validation sets are sorted by
(increasing) duration before the start of the first epoch (default:
\texttt{False}).

The \texttt{train\_model} function defaults to using spectrogram
features; if you choose to use these features, note that the acoustic
model in \texttt{simple\_rnn\_model} should have
\texttt{input\_dim=161}. Otherwise, if you choose to use MFCC features,
the acoustic model should have \texttt{input\_dim=13}.

We have chosen to use \texttt{GRU} units in the supplied RNN. If you
would like to experiment with \texttt{LSTM} or \texttt{SimpleRNN} cells,
feel free to do so here. If you change the \texttt{GRU} units to
\texttt{SimpleRNN} cells in \texttt{simple\_rnn\_model}, you may notice
that the loss quickly becomes undefined (\texttt{nan}) - you are
strongly encouraged to check this for yourself! This is due to the
\href{http://www.wildml.com/2015/10/recurrent-neural-networks-tutorial-part-3-backpropagation-through-time-and-vanishing-gradients/}{exploding
gradients problem}. We have already implemented
\href{https://arxiv.org/pdf/1211.5063.pdf}{gradient clipping} in your
optimizer to help you avoid this issue.

\textbf{IMPORTANT NOTE:} If you notice that your gradient has exploded
in any of the models below, feel free to explore more with gradient
clipping (the \texttt{clipnorm} argument in your optimizer) or swap out
any \texttt{SimpleRNN} cells for \texttt{LSTM} or \texttt{GRU} cells.
You can also try restarting the kernel to restart the training process.

    \begin{Verbatim}[commandchars=\\\{\}]
{\color{incolor}In [{\color{incolor}7}]:} \PY{n}{train\PYZus{}model}\PY{p}{(}\PY{n}{input\PYZus{}to\PYZus{}softmax}\PY{o}{=}\PY{n}{model\PYZus{}0}\PY{p}{,} 
                    \PY{n}{pickle\PYZus{}path}\PY{o}{=}\PY{l+s+s1}{\PYZsq{}}\PY{l+s+s1}{model\PYZus{}0.pickle}\PY{l+s+s1}{\PYZsq{}}\PY{p}{,} 
                    \PY{n}{save\PYZus{}model\PYZus{}path}\PY{o}{=}\PY{l+s+s1}{\PYZsq{}}\PY{l+s+s1}{model\PYZus{}0.h5}\PY{l+s+s1}{\PYZsq{}}\PY{p}{,}
                    \PY{n}{spectrogram}\PY{o}{=}\PY{k+kc}{True}\PY{p}{)} \PY{c+c1}{\PYZsh{} change to False if you would like to use MFCC features}
\end{Verbatim}


    \begin{Verbatim}[commandchars=\\\{\}]
Epoch 1/20
101/101 [==============================] - 121s - loss: 875.0425 - val\_loss: 759.6363
Epoch 2/20
101/101 [==============================] - 121s - loss: 781.7915 - val\_loss: 762.9231
Epoch 3/20
101/101 [==============================] - 122s - loss: 779.6952 - val\_loss: 752.8900
Epoch 4/20
101/101 [==============================] - 122s - loss: 779.8681 - val\_loss: 766.0658
Epoch 5/20
101/101 [==============================] - 120s - loss: 779.4564 - val\_loss: 743.3448
Epoch 6/20
101/101 [==============================] - 122s - loss: 778.1595 - val\_loss: 751.8000
Epoch 7/20
101/101 [==============================] - 121s - loss: 777.6926 - val\_loss: 757.8558
Epoch 8/20
101/101 [==============================] - 121s - loss: 778.0011 - val\_loss: 762.1862
Epoch 9/20
101/101 [==============================] - 121s - loss: 777.4321 - val\_loss: 755.9904
Epoch 10/20
101/101 [==============================] - 121s - loss: 778.2705 - val\_loss: 759.0700
Epoch 11/20
101/101 [==============================] - 121s - loss: 777.5723 - val\_loss: 751.0261
Epoch 12/20
101/101 [==============================] - 121s - loss: 777.7709 - val\_loss: 759.9874
Epoch 13/20
101/101 [==============================] - 122s - loss: 777.9392 - val\_loss: 752.9891
Epoch 14/20
101/101 [==============================] - 121s - loss: 777.9915 - val\_loss: 759.5672
Epoch 15/20
101/101 [==============================] - 121s - loss: 777.6391 - val\_loss: 760.2755
Epoch 16/20
101/101 [==============================] - 122s - loss: 777.7038 - val\_loss: 752.7277
Epoch 17/20
101/101 [==============================] - 121s - loss: 777.6779 - val\_loss: 757.6810
Epoch 18/20
101/101 [==============================] - 122s - loss: 777.6803 - val\_loss: 751.7398
Epoch 19/20
101/101 [==============================] - 122s - loss: 777.3186 - val\_loss: 756.5977
Epoch 20/20
101/101 [==============================] - 121s - loss: 777.5161 - val\_loss: 755.7074

    \end{Verbatim}

     \#\#\# (IMPLEMENTATION) Model 1: RNN + TimeDistributed Dense

Read about the \href{https://keras.io/layers/wrappers/}{TimeDistributed}
wrapper and the
\href{https://keras.io/layers/normalization/}{BatchNormalization} layer
in the Keras documentation. For your next architecture, you will add
\href{https://arxiv.org/pdf/1510.01378.pdf}{batch normalization} to the
recurrent layer to reduce training times. The \texttt{TimeDistributed}
layer will be used to find more complex patterns in the dataset. The
unrolled snapshot of the architecture is depicted below.

The next figure shows an equivalent, rolled depiction of the RNN that
shows the (\texttt{TimeDistrbuted}) dense and output layers in greater
detail.

Use your research to complete the \texttt{rnn\_model} function within
the \texttt{sample\_models.py} file. The function should specify an
architecture that satisfies the following requirements: - The first
layer of the neural network should be an RNN (\texttt{SimpleRNN},
\texttt{LSTM}, or \texttt{GRU}) that takes the time sequence of audio
features as input. We have added \texttt{GRU} units for you, but feel
free to change \texttt{GRU} to \texttt{SimpleRNN} or \texttt{LSTM}, if
you like! - Whereas the architecture in \texttt{simple\_rnn\_model}
treated the RNN output as the final layer of the model, you will use the
output of your RNN as a hidden layer. Use \texttt{TimeDistributed} to
apply a \texttt{Dense} layer to each of the time steps in the RNN
output. Ensure that each \texttt{Dense} layer has \texttt{output\_dim}
units.

Use the code cell below to load your model into the \texttt{model\_1}
variable. Use a value for \texttt{input\_dim} that matches your chosen
audio features, and feel free to change the values for \texttt{units}
and \texttt{activation} to tweak the behavior of your recurrent layer.

    \begin{Verbatim}[commandchars=\\\{\}]
{\color{incolor}In [{\color{incolor}7}]:} \PY{n}{model\PYZus{}1} \PY{o}{=} \PY{n}{rnn\PYZus{}model}\PY{p}{(}\PY{n}{input\PYZus{}dim}\PY{o}{=}\PY{l+m+mi}{161}\PY{p}{,} \PY{c+c1}{\PYZsh{} change to 13 if you would like to use MFCC features}
                            \PY{n}{units}\PY{o}{=}\PY{l+m+mi}{200}\PY{p}{,}
                            \PY{n}{activation}\PY{o}{=}\PY{l+s+s1}{\PYZsq{}}\PY{l+s+s1}{relu}\PY{l+s+s1}{\PYZsq{}}\PY{p}{)}
\end{Verbatim}


    \begin{Verbatim}[commandchars=\\\{\}]
\_\_\_\_\_\_\_\_\_\_\_\_\_\_\_\_\_\_\_\_\_\_\_\_\_\_\_\_\_\_\_\_\_\_\_\_\_\_\_\_\_\_\_\_\_\_\_\_\_\_\_\_\_\_\_\_\_\_\_\_\_\_\_\_\_
Layer (type)                 Output Shape              Param \#   
=================================================================
the\_input (InputLayer)       (None, None, 161)         0         
\_\_\_\_\_\_\_\_\_\_\_\_\_\_\_\_\_\_\_\_\_\_\_\_\_\_\_\_\_\_\_\_\_\_\_\_\_\_\_\_\_\_\_\_\_\_\_\_\_\_\_\_\_\_\_\_\_\_\_\_\_\_\_\_\_
rnn (GRU)                    (None, None, 200)         217200    
\_\_\_\_\_\_\_\_\_\_\_\_\_\_\_\_\_\_\_\_\_\_\_\_\_\_\_\_\_\_\_\_\_\_\_\_\_\_\_\_\_\_\_\_\_\_\_\_\_\_\_\_\_\_\_\_\_\_\_\_\_\_\_\_\_
batch\_normalization\_1 (Batch (None, None, 200)         800       
\_\_\_\_\_\_\_\_\_\_\_\_\_\_\_\_\_\_\_\_\_\_\_\_\_\_\_\_\_\_\_\_\_\_\_\_\_\_\_\_\_\_\_\_\_\_\_\_\_\_\_\_\_\_\_\_\_\_\_\_\_\_\_\_\_
time\_distributed\_1 (TimeDist (None, None, 29)          5829      
\_\_\_\_\_\_\_\_\_\_\_\_\_\_\_\_\_\_\_\_\_\_\_\_\_\_\_\_\_\_\_\_\_\_\_\_\_\_\_\_\_\_\_\_\_\_\_\_\_\_\_\_\_\_\_\_\_\_\_\_\_\_\_\_\_
softmax (Activation)         (None, None, 29)          0         
=================================================================
Total params: 223,829
Trainable params: 223,429
Non-trainable params: 400
\_\_\_\_\_\_\_\_\_\_\_\_\_\_\_\_\_\_\_\_\_\_\_\_\_\_\_\_\_\_\_\_\_\_\_\_\_\_\_\_\_\_\_\_\_\_\_\_\_\_\_\_\_\_\_\_\_\_\_\_\_\_\_\_\_
None

    \end{Verbatim}

    Please execute the code cell below to train the neural network you
specified in \texttt{input\_to\_softmax}. After the model has finished
training, the model is
\href{https://keras.io/getting-started/faq/\#how-can-i-save-a-keras-model}{saved}
in the HDF5 file \texttt{model\_1.h5}. The loss history is
\href{https://wiki.python.org/moin/UsingPickle}{saved} in
\texttt{model\_1.pickle}. You are welcome to tweak any of the optional
parameters while calling the \texttt{train\_model} function, but this is
not required.

    \begin{Verbatim}[commandchars=\\\{\}]
{\color{incolor}In [{\color{incolor}8}]:} \PY{n}{train\PYZus{}model}\PY{p}{(}\PY{n}{input\PYZus{}to\PYZus{}softmax}\PY{o}{=}\PY{n}{model\PYZus{}1}\PY{p}{,} 
                    \PY{n}{pickle\PYZus{}path}\PY{o}{=}\PY{l+s+s1}{\PYZsq{}}\PY{l+s+s1}{model\PYZus{}1.pickle}\PY{l+s+s1}{\PYZsq{}}\PY{p}{,} 
                    \PY{n}{save\PYZus{}model\PYZus{}path}\PY{o}{=}\PY{l+s+s1}{\PYZsq{}}\PY{l+s+s1}{model\PYZus{}1.h5}\PY{l+s+s1}{\PYZsq{}}\PY{p}{,}
                    \PY{n}{spectrogram}\PY{o}{=}\PY{k+kc}{True}\PY{p}{)} \PY{c+c1}{\PYZsh{} change to False if you would like to use MFCC features}
\end{Verbatim}


    \begin{Verbatim}[commandchars=\\\{\}]
Epoch 1/20
101/101 [==============================] - 130s - loss: 294.0972 - val\_loss: 234.5631
Epoch 2/20
101/101 [==============================] - 130s - loss: 222.7050 - val\_loss: 215.9393
Epoch 3/20
101/101 [==============================] - 136s - loss: 199.5644 - val\_loss: 189.4452
Epoch 4/20
101/101 [==============================] - 136s - loss: 180.0095 - val\_loss: 172.5355
Epoch 5/20
101/101 [==============================] - 139s - loss: 165.6800 - val\_loss: 162.9439
Epoch 6/20
101/101 [==============================] - 128s - loss: 155.8112 - val\_loss: 159.6140
Epoch 7/20
101/101 [==============================] - 127s - loss: 150.1081 - val\_loss: 154.4863
Epoch 8/20
101/101 [==============================] - 128s - loss: 145.7455 - val\_loss: 153.3493
Epoch 9/20
101/101 [==============================] - 128s - loss: 141.8987 - val\_loss: 148.2581
Epoch 10/20
101/101 [==============================] - 128s - loss: 139.2543 - val\_loss: 149.2268
Epoch 11/20
101/101 [==============================] - 128s - loss: 138.5021 - val\_loss: 151.4975
Epoch 12/20
101/101 [==============================] - 129s - loss: 138.9227 - val\_loss: 150.6742
Epoch 13/20
101/101 [==============================] - 129s - loss: 137.6266 - val\_loss: 147.5385
Epoch 14/20
101/101 [==============================] - 129s - loss: 133.4913 - val\_loss: 148.6309
Epoch 15/20
101/101 [==============================] - 129s - loss: 133.5592 - val\_loss: 149.5861
Epoch 16/20
101/101 [==============================] - 129s - loss: 135.8490 - val\_loss: 148.7486
Epoch 17/20
101/101 [==============================] - 127s - loss: 135.4782 - val\_loss: 152.5208
Epoch 18/20
101/101 [==============================] - 129s - loss: 134.7642 - val\_loss: 147.1323
Epoch 19/20
101/101 [==============================] - 127s - loss: 134.0606 - val\_loss: 148.5466
Epoch 20/20
101/101 [==============================] - 127s - loss: 132.8362 - val\_loss: 143.8978

    \end{Verbatim}

     \#\#\# (IMPLEMENTATION) Model 2: CNN + RNN + TimeDistributed Dense

The architecture in \texttt{cnn\_rnn\_model} adds an additional level of
complexity, by introducing a
\href{https://keras.io/layers/convolutional/\#conv1d}{1D convolution
layer}.

This layer incorporates many arguments that can be (optionally) tuned
when calling the \texttt{cnn\_rnn\_model} module. We provide sample
starting parameters, which you might find useful if you choose to use
spectrogram audio features.

If you instead want to use MFCC features, these arguments will have to
be tuned. Note that the current architecture only supports values of
\texttt{\textquotesingle{}same\textquotesingle{}} or
\texttt{\textquotesingle{}valid\textquotesingle{}} for the
\texttt{conv\_border\_mode} argument.

When tuning the parameters, be careful not to choose settings that make
the convolutional layer overly small. If the temporal length of the CNN
layer is shorter than the length of the transcribed text label, your
code will throw an error.

Before running the code cell below, you must modify the
\texttt{cnn\_rnn\_model} function in \texttt{sample\_models.py}. Please
add batch normalization to the recurrent layer, and provide the same
\texttt{TimeDistributed} layer as before.

    \begin{Verbatim}[commandchars=\\\{\}]
{\color{incolor}In [{\color{incolor}9}]:} \PY{n}{model\PYZus{}2} \PY{o}{=} \PY{n}{cnn\PYZus{}rnn\PYZus{}model}\PY{p}{(}\PY{n}{input\PYZus{}dim}\PY{o}{=}\PY{l+m+mi}{161}\PY{p}{,} \PY{c+c1}{\PYZsh{} change to 13 if you would like to use MFCC features}
                                \PY{n}{filters}\PY{o}{=}\PY{l+m+mi}{200}\PY{p}{,}
                                \PY{n}{kernel\PYZus{}size}\PY{o}{=}\PY{l+m+mi}{11}\PY{p}{,} 
                                \PY{n}{conv\PYZus{}stride}\PY{o}{=}\PY{l+m+mi}{2}\PY{p}{,}
                                \PY{n}{conv\PYZus{}border\PYZus{}mode}\PY{o}{=}\PY{l+s+s1}{\PYZsq{}}\PY{l+s+s1}{valid}\PY{l+s+s1}{\PYZsq{}}\PY{p}{,}
                                \PY{n}{units}\PY{o}{=}\PY{l+m+mi}{200}\PY{p}{)}
\end{Verbatim}


    \begin{Verbatim}[commandchars=\\\{\}]
\_\_\_\_\_\_\_\_\_\_\_\_\_\_\_\_\_\_\_\_\_\_\_\_\_\_\_\_\_\_\_\_\_\_\_\_\_\_\_\_\_\_\_\_\_\_\_\_\_\_\_\_\_\_\_\_\_\_\_\_\_\_\_\_\_
Layer (type)                 Output Shape              Param \#   
=================================================================
the\_input (InputLayer)       (None, None, 161)         0         
\_\_\_\_\_\_\_\_\_\_\_\_\_\_\_\_\_\_\_\_\_\_\_\_\_\_\_\_\_\_\_\_\_\_\_\_\_\_\_\_\_\_\_\_\_\_\_\_\_\_\_\_\_\_\_\_\_\_\_\_\_\_\_\_\_
conv1d (Conv1D)              (None, None, 200)         354400    
\_\_\_\_\_\_\_\_\_\_\_\_\_\_\_\_\_\_\_\_\_\_\_\_\_\_\_\_\_\_\_\_\_\_\_\_\_\_\_\_\_\_\_\_\_\_\_\_\_\_\_\_\_\_\_\_\_\_\_\_\_\_\_\_\_
bn\_conv\_1d (BatchNormalizati (None, None, 200)         800       
\_\_\_\_\_\_\_\_\_\_\_\_\_\_\_\_\_\_\_\_\_\_\_\_\_\_\_\_\_\_\_\_\_\_\_\_\_\_\_\_\_\_\_\_\_\_\_\_\_\_\_\_\_\_\_\_\_\_\_\_\_\_\_\_\_
rnn (SimpleRNN)              (None, None, 200)         80200     
\_\_\_\_\_\_\_\_\_\_\_\_\_\_\_\_\_\_\_\_\_\_\_\_\_\_\_\_\_\_\_\_\_\_\_\_\_\_\_\_\_\_\_\_\_\_\_\_\_\_\_\_\_\_\_\_\_\_\_\_\_\_\_\_\_
batch\_normalization\_2 (Batch (None, None, 200)         800       
\_\_\_\_\_\_\_\_\_\_\_\_\_\_\_\_\_\_\_\_\_\_\_\_\_\_\_\_\_\_\_\_\_\_\_\_\_\_\_\_\_\_\_\_\_\_\_\_\_\_\_\_\_\_\_\_\_\_\_\_\_\_\_\_\_
time\_distributed\_2 (TimeDist (None, None, 29)          5829      
\_\_\_\_\_\_\_\_\_\_\_\_\_\_\_\_\_\_\_\_\_\_\_\_\_\_\_\_\_\_\_\_\_\_\_\_\_\_\_\_\_\_\_\_\_\_\_\_\_\_\_\_\_\_\_\_\_\_\_\_\_\_\_\_\_
softmax (Activation)         (None, None, 29)          0         
=================================================================
Total params: 442,029
Trainable params: 441,229
Non-trainable params: 800
\_\_\_\_\_\_\_\_\_\_\_\_\_\_\_\_\_\_\_\_\_\_\_\_\_\_\_\_\_\_\_\_\_\_\_\_\_\_\_\_\_\_\_\_\_\_\_\_\_\_\_\_\_\_\_\_\_\_\_\_\_\_\_\_\_
None

    \end{Verbatim}

    Please execute the code cell below to train the neural network you
specified in \texttt{input\_to\_softmax}. After the model has finished
training, the model is
\href{https://keras.io/getting-started/faq/\#how-can-i-save-a-keras-model}{saved}
in the HDF5 file \texttt{model\_2.h5}. The loss history is
\href{https://wiki.python.org/moin/UsingPickle}{saved} in
\texttt{model\_2.pickle}. You are welcome to tweak any of the optional
parameters while calling the \texttt{train\_model} function, but this is
not required.

    \begin{Verbatim}[commandchars=\\\{\}]
{\color{incolor}In [{\color{incolor}10}]:} \PY{n}{train\PYZus{}model}\PY{p}{(}\PY{n}{input\PYZus{}to\PYZus{}softmax}\PY{o}{=}\PY{n}{model\PYZus{}2}\PY{p}{,} 
                     \PY{n}{pickle\PYZus{}path}\PY{o}{=}\PY{l+s+s1}{\PYZsq{}}\PY{l+s+s1}{model\PYZus{}2.pickle}\PY{l+s+s1}{\PYZsq{}}\PY{p}{,} 
                     \PY{n}{save\PYZus{}model\PYZus{}path}\PY{o}{=}\PY{l+s+s1}{\PYZsq{}}\PY{l+s+s1}{model\PYZus{}2.h5}\PY{l+s+s1}{\PYZsq{}}\PY{p}{,} 
                     \PY{n}{spectrogram}\PY{o}{=}\PY{k+kc}{True}\PY{p}{)} \PY{c+c1}{\PYZsh{} change to False if you would like to use MFCC features}
\end{Verbatim}


    \begin{Verbatim}[commandchars=\\\{\}]
Epoch 1/20
101/101 [==============================] - 33s - loss: 252.8165 - val\_loss: 222.6869
Epoch 2/20
101/101 [==============================] - 30s - loss: 201.9033 - val\_loss: 186.7880
Epoch 3/20
101/101 [==============================] - 30s - loss: 167.1468 - val\_loss: 165.2882
Epoch 4/20
101/101 [==============================] - 30s - loss: 150.0536 - val\_loss: 150.6287
Epoch 5/20
101/101 [==============================] - 30s - loss: 139.7259 - val\_loss: 144.9637
Epoch 6/20
101/101 [==============================] - 30s - loss: 132.0942 - val\_loss: 142.5094
Epoch 7/20
101/101 [==============================] - 30s - loss: 125.9157 - val\_loss: 140.1113
Epoch 8/20
101/101 [==============================] - 30s - loss: 120.5592 - val\_loss: 138.4688
Epoch 9/20
101/101 [==============================] - 30s - loss: 116.8739 - val\_loss: 138.0747
Epoch 10/20
101/101 [==============================] - 30s - loss: 112.7829 - val\_loss: 138.8593
Epoch 11/20
101/101 [==============================] - 30s - loss: 108.8849 - val\_loss: 138.8298
Epoch 12/20
101/101 [==============================] - 31s - loss: 105.6637 - val\_loss: 137.6842
Epoch 13/20
101/101 [==============================] - 30s - loss: 102.5688 - val\_loss: 138.3234
Epoch 14/20
101/101 [==============================] - 30s - loss: 99.4097 - val\_loss: 139.6775
Epoch 15/20
101/101 [==============================] - 31s - loss: 96.3642 - val\_loss: 139.2303
Epoch 16/20
101/101 [==============================] - 30s - loss: 93.7236 - val\_loss: 141.6438
Epoch 17/20
101/101 [==============================] - 30s - loss: 91.0387 - val\_loss: 141.6714
Epoch 18/20
101/101 [==============================] - 30s - loss: 88.6905 - val\_loss: 143.3218
Epoch 19/20
101/101 [==============================] - 30s - loss: 86.3790 - val\_loss: 146.9814
Epoch 20/20
101/101 [==============================] - 30s - loss: 83.8334 - val\_loss: 146.2520

    \end{Verbatim}

     \#\#\# (IMPLEMENTATION) Model 3: Deeper RNN + TimeDistributed Dense

Review the code in \texttt{rnn\_model}, which makes use of a single
recurrent layer. Now, specify an architecture in
\texttt{deep\_rnn\_model} that utilizes a variable number
\texttt{recur\_layers} of recurrent layers. The figure below shows the
architecture that should be returned if \texttt{recur\_layers=2}. In the
figure, the output sequence of the first recurrent layer is used as
input for the next recurrent layer.

Feel free to change the supplied values of \texttt{units} to whatever
you think performs best. You can change the value of
\texttt{recur\_layers}, as long as your final value is greater than 1.
(As a quick check that you have implemented the additional functionality
in \texttt{deep\_rnn\_model} correctly, make sure that the architecture
that you specify here is identical to \texttt{rnn\_model} if
\texttt{recur\_layers=1}.)

    \begin{Verbatim}[commandchars=\\\{\}]
{\color{incolor}In [{\color{incolor}12}]:} \PY{n}{model\PYZus{}3} \PY{o}{=} \PY{n}{deep\PYZus{}rnn\PYZus{}model}\PY{p}{(}\PY{n}{input\PYZus{}dim}\PY{o}{=}\PY{l+m+mi}{161}\PY{p}{,} \PY{c+c1}{\PYZsh{} change to 13 if you would like to use MFCC features}
                                  \PY{n}{units}\PY{o}{=}\PY{l+m+mi}{200}\PY{p}{,}
                                  \PY{n}{recur\PYZus{}layers}\PY{o}{=}\PY{l+m+mi}{2}\PY{p}{)} 
\end{Verbatim}


    \begin{Verbatim}[commandchars=\\\{\}]
\_\_\_\_\_\_\_\_\_\_\_\_\_\_\_\_\_\_\_\_\_\_\_\_\_\_\_\_\_\_\_\_\_\_\_\_\_\_\_\_\_\_\_\_\_\_\_\_\_\_\_\_\_\_\_\_\_\_\_\_\_\_\_\_\_
Layer (type)                 Output Shape              Param \#   
=================================================================
the\_input (InputLayer)       (None, None, 161)         0         
\_\_\_\_\_\_\_\_\_\_\_\_\_\_\_\_\_\_\_\_\_\_\_\_\_\_\_\_\_\_\_\_\_\_\_\_\_\_\_\_\_\_\_\_\_\_\_\_\_\_\_\_\_\_\_\_\_\_\_\_\_\_\_\_\_
rnn (GRU)                    (None, None, 200)         217200    
\_\_\_\_\_\_\_\_\_\_\_\_\_\_\_\_\_\_\_\_\_\_\_\_\_\_\_\_\_\_\_\_\_\_\_\_\_\_\_\_\_\_\_\_\_\_\_\_\_\_\_\_\_\_\_\_\_\_\_\_\_\_\_\_\_
batch\_normalization\_6 (Batch (None, None, 200)         800       
\_\_\_\_\_\_\_\_\_\_\_\_\_\_\_\_\_\_\_\_\_\_\_\_\_\_\_\_\_\_\_\_\_\_\_\_\_\_\_\_\_\_\_\_\_\_\_\_\_\_\_\_\_\_\_\_\_\_\_\_\_\_\_\_\_
gru\_3 (GRU)                  (None, None, 200)         240600    
\_\_\_\_\_\_\_\_\_\_\_\_\_\_\_\_\_\_\_\_\_\_\_\_\_\_\_\_\_\_\_\_\_\_\_\_\_\_\_\_\_\_\_\_\_\_\_\_\_\_\_\_\_\_\_\_\_\_\_\_\_\_\_\_\_
batch\_normalization\_7 (Batch (None, None, 200)         800       
\_\_\_\_\_\_\_\_\_\_\_\_\_\_\_\_\_\_\_\_\_\_\_\_\_\_\_\_\_\_\_\_\_\_\_\_\_\_\_\_\_\_\_\_\_\_\_\_\_\_\_\_\_\_\_\_\_\_\_\_\_\_\_\_\_
gru\_4 (GRU)                  (None, None, 200)         240600    
\_\_\_\_\_\_\_\_\_\_\_\_\_\_\_\_\_\_\_\_\_\_\_\_\_\_\_\_\_\_\_\_\_\_\_\_\_\_\_\_\_\_\_\_\_\_\_\_\_\_\_\_\_\_\_\_\_\_\_\_\_\_\_\_\_
batch\_normalization\_8 (Batch (None, None, 200)         800       
\_\_\_\_\_\_\_\_\_\_\_\_\_\_\_\_\_\_\_\_\_\_\_\_\_\_\_\_\_\_\_\_\_\_\_\_\_\_\_\_\_\_\_\_\_\_\_\_\_\_\_\_\_\_\_\_\_\_\_\_\_\_\_\_\_
time\_distributed\_4 (TimeDist (None, None, 29)          5829      
\_\_\_\_\_\_\_\_\_\_\_\_\_\_\_\_\_\_\_\_\_\_\_\_\_\_\_\_\_\_\_\_\_\_\_\_\_\_\_\_\_\_\_\_\_\_\_\_\_\_\_\_\_\_\_\_\_\_\_\_\_\_\_\_\_
softmax (Activation)         (None, None, 29)          0         
=================================================================
Total params: 706,629
Trainable params: 705,429
Non-trainable params: 1,200
\_\_\_\_\_\_\_\_\_\_\_\_\_\_\_\_\_\_\_\_\_\_\_\_\_\_\_\_\_\_\_\_\_\_\_\_\_\_\_\_\_\_\_\_\_\_\_\_\_\_\_\_\_\_\_\_\_\_\_\_\_\_\_\_\_
None

    \end{Verbatim}

    Please execute the code cell below to train the neural network you
specified in \texttt{input\_to\_softmax}. After the model has finished
training, the model is
\href{https://keras.io/getting-started/faq/\#how-can-i-save-a-keras-model}{saved}
in the HDF5 file \texttt{model\_3.h5}. The loss history is
\href{https://wiki.python.org/moin/UsingPickle}{saved} in
\texttt{model\_3.pickle}. You are welcome to tweak any of the optional
parameters while calling the \texttt{train\_model} function, but this is
not required.

    \begin{Verbatim}[commandchars=\\\{\}]
{\color{incolor}In [{\color{incolor}13}]:} \PY{n}{train\PYZus{}model}\PY{p}{(}\PY{n}{input\PYZus{}to\PYZus{}softmax}\PY{o}{=}\PY{n}{model\PYZus{}3}\PY{p}{,} 
                     \PY{n}{pickle\PYZus{}path}\PY{o}{=}\PY{l+s+s1}{\PYZsq{}}\PY{l+s+s1}{model\PYZus{}3.pickle}\PY{l+s+s1}{\PYZsq{}}\PY{p}{,} 
                     \PY{n}{save\PYZus{}model\PYZus{}path}\PY{o}{=}\PY{l+s+s1}{\PYZsq{}}\PY{l+s+s1}{model\PYZus{}3.h5}\PY{l+s+s1}{\PYZsq{}}\PY{p}{,} 
                     \PY{n}{spectrogram}\PY{o}{=}\PY{k+kc}{True}\PY{p}{)} \PY{c+c1}{\PYZsh{} change to False if you would like to use MFCC features}
\end{Verbatim}


    \begin{Verbatim}[commandchars=\\\{\}]
Epoch 1/20
101/101 [==============================] - 391s - loss: 378.4320 - val\_loss: 350.0452
Epoch 2/20
101/101 [==============================] - 381s - loss: 293.9654 - val\_loss: 307.2442
Epoch 3/20
101/101 [==============================] - 383s - loss: 249.6412 - val\_loss: 252.0128
Epoch 4/20
101/101 [==============================] - 376s - loss: 220.9429 - val\_loss: 246.3732
Epoch 5/20
101/101 [==============================] - 397s - loss: 184.3556 - val\_loss: 183.7175
Epoch 6/20
101/101 [==============================] - 412s - loss: 160.3390 - val\_loss: 164.3829
Epoch 7/20
101/101 [==============================] - 398s - loss: 145.5413 - val\_loss: 152.8557
Epoch 8/20
101/101 [==============================] - 402s - loss: 136.7868 - val\_loss: 149.0385
Epoch 9/20
101/101 [==============================] - 407s - loss: 130.6174 - val\_loss: 144.6200
Epoch 10/20
101/101 [==============================] - 424s - loss: 125.4280 - val\_loss: 148.2122
Epoch 11/20
101/101 [==============================] - 411s - loss: 121.2760 - val\_loss: 140.4631
Epoch 12/20
101/101 [==============================] - 409s - loss: 118.8468 - val\_loss: 142.8345
Epoch 13/20
101/101 [==============================] - 423s - loss: 115.1601 - val\_loss: 134.6761
Epoch 14/20
101/101 [==============================] - 418s - loss: 113.6064 - val\_loss: 131.3338
Epoch 15/20
101/101 [==============================] - 416s - loss: 110.0106 - val\_loss: 134.0708
Epoch 16/20
101/101 [==============================] - 420s - loss: 108.6577 - val\_loss: 130.3048
Epoch 17/20
101/101 [==============================] - 417s - loss: 106.2128 - val\_loss: 129.0051
Epoch 18/20
101/101 [==============================] - 418s - loss: 104.5341 - val\_loss: 130.7819
Epoch 19/20
101/101 [==============================] - 417s - loss: 105.2835 - val\_loss: 128.8108
Epoch 20/20
101/101 [==============================] - 382s - loss: 103.1495 - val\_loss: 131.5369

    \end{Verbatim}

     \#\#\# (IMPLEMENTATION) Model 4: Bidirectional RNN + TimeDistributed
Dense

Read about the \href{https://keras.io/layers/wrappers/}{Bidirectional}
wrapper in the Keras documentation. For your next architecture, you will
specify an architecture that uses a single bidirectional RNN layer,
before a (\texttt{TimeDistributed}) dense layer. The added value of a
bidirectional RNN is described well in
\href{http://www.cs.toronto.edu/~hinton/absps/DRNN_speech.pdf}{this
paper}. \textgreater{} One shortcoming of conventional RNNs is that they
are only able to make use of previous context. In speech recognition,
where whole utterances are transcribed at once, there is no reason not
to exploit future context as well. Bidirectional RNNs (BRNNs) do this by
processing the data in both directions with two separate hidden layers
which are then fed forwards to the same output layer.

Before running the code cell below, you must complete the
\texttt{bidirectional\_rnn\_model} function in
\texttt{sample\_models.py}. Feel free to use \texttt{SimpleRNN},
\texttt{LSTM}, or \texttt{GRU} units. When specifying the
\texttt{Bidirectional} wrapper, use
\texttt{merge\_mode=\textquotesingle{}concat\textquotesingle{}}.

    \begin{Verbatim}[commandchars=\\\{\}]
{\color{incolor}In [{\color{incolor}14}]:} \PY{n}{model\PYZus{}4} \PY{o}{=} \PY{n}{bidirectional\PYZus{}rnn\PYZus{}model}\PY{p}{(}\PY{n}{input\PYZus{}dim}\PY{o}{=}\PY{l+m+mi}{161}\PY{p}{,} \PY{c+c1}{\PYZsh{} change to 13 if you would like to use MFCC features}
                                           \PY{n}{units}\PY{o}{=}\PY{l+m+mi}{200}\PY{p}{)}
\end{Verbatim}


    \begin{Verbatim}[commandchars=\\\{\}]
\_\_\_\_\_\_\_\_\_\_\_\_\_\_\_\_\_\_\_\_\_\_\_\_\_\_\_\_\_\_\_\_\_\_\_\_\_\_\_\_\_\_\_\_\_\_\_\_\_\_\_\_\_\_\_\_\_\_\_\_\_\_\_\_\_
Layer (type)                 Output Shape              Param \#   
=================================================================
the\_input (InputLayer)       (None, None, 161)         0         
\_\_\_\_\_\_\_\_\_\_\_\_\_\_\_\_\_\_\_\_\_\_\_\_\_\_\_\_\_\_\_\_\_\_\_\_\_\_\_\_\_\_\_\_\_\_\_\_\_\_\_\_\_\_\_\_\_\_\_\_\_\_\_\_\_
bidirectional\_1 (Bidirection (None, None, 400)         434400    
\_\_\_\_\_\_\_\_\_\_\_\_\_\_\_\_\_\_\_\_\_\_\_\_\_\_\_\_\_\_\_\_\_\_\_\_\_\_\_\_\_\_\_\_\_\_\_\_\_\_\_\_\_\_\_\_\_\_\_\_\_\_\_\_\_
time\_distributed\_5 (TimeDist (None, None, 29)          11629     
\_\_\_\_\_\_\_\_\_\_\_\_\_\_\_\_\_\_\_\_\_\_\_\_\_\_\_\_\_\_\_\_\_\_\_\_\_\_\_\_\_\_\_\_\_\_\_\_\_\_\_\_\_\_\_\_\_\_\_\_\_\_\_\_\_
softmax (Activation)         (None, None, 29)          0         
=================================================================
Total params: 446,029
Trainable params: 446,029
Non-trainable params: 0
\_\_\_\_\_\_\_\_\_\_\_\_\_\_\_\_\_\_\_\_\_\_\_\_\_\_\_\_\_\_\_\_\_\_\_\_\_\_\_\_\_\_\_\_\_\_\_\_\_\_\_\_\_\_\_\_\_\_\_\_\_\_\_\_\_
None

    \end{Verbatim}

    Please execute the code cell below to train the neural network you
specified in \texttt{input\_to\_softmax}. After the model has finished
training, the model is
\href{https://keras.io/getting-started/faq/\#how-can-i-save-a-keras-model}{saved}
in the HDF5 file \texttt{model\_4.h5}. The loss history is
\href{https://wiki.python.org/moin/UsingPickle}{saved} in
\texttt{model\_4.pickle}. You are welcome to tweak any of the optional
parameters while calling the \texttt{train\_model} function, but this is
not required.

    \begin{Verbatim}[commandchars=\\\{\}]
{\color{incolor}In [{\color{incolor}15}]:} \PY{n}{train\PYZus{}model}\PY{p}{(}\PY{n}{input\PYZus{}to\PYZus{}softmax}\PY{o}{=}\PY{n}{model\PYZus{}4}\PY{p}{,} 
                     \PY{n}{pickle\PYZus{}path}\PY{o}{=}\PY{l+s+s1}{\PYZsq{}}\PY{l+s+s1}{model\PYZus{}4.pickle}\PY{l+s+s1}{\PYZsq{}}\PY{p}{,} 
                     \PY{n}{save\PYZus{}model\PYZus{}path}\PY{o}{=}\PY{l+s+s1}{\PYZsq{}}\PY{l+s+s1}{model\PYZus{}4.h5}\PY{l+s+s1}{\PYZsq{}}\PY{p}{,} 
                     \PY{n}{spectrogram}\PY{o}{=}\PY{k+kc}{True}\PY{p}{)} \PY{c+c1}{\PYZsh{} change to False if you would like to use MFCC features}
\end{Verbatim}


    \begin{Verbatim}[commandchars=\\\{\}]
Epoch 1/20
101/101 [==============================] - 259s - loss: 279.9830 - val\_loss: 222.8155
Epoch 2/20
101/101 [==============================] - 265s - loss: 218.2213 - val\_loss: 201.6204
Epoch 3/20
101/101 [==============================] - 265s - loss: 201.6461 - val\_loss: 192.6017
Epoch 4/20
101/101 [==============================] - 268s - loss: 189.6004 - val\_loss: 180.0925
Epoch 5/20
101/101 [==============================] - 268s - loss: 179.4330 - val\_loss: 175.7925
Epoch 6/20
101/101 [==============================] - 265s - loss: 170.9792 - val\_loss: 167.8010
Epoch 7/20
101/101 [==============================] - 274s - loss: 163.7791 - val\_loss: 168.4055
Epoch 8/20
101/101 [==============================] - 275s - loss: 157.3778 - val\_loss: 158.0000
Epoch 9/20
101/101 [==============================] - 283s - loss: 151.9820 - val\_loss: 159.4304
Epoch 10/20
101/101 [==============================] - 274s - loss: 146.6211 - val\_loss: 153.3350
Epoch 11/20
101/101 [==============================] - 277s - loss: 142.2848 - val\_loss: 154.4290
Epoch 12/20
101/101 [==============================] - 278s - loss: 137.9305 - val\_loss: 149.7766
Epoch 13/20
101/101 [==============================] - 278s - loss: 134.1649 - val\_loss: 148.7869
Epoch 14/20
101/101 [==============================] - 274s - loss: 130.4130 - val\_loss: 146.3070
Epoch 15/20
101/101 [==============================] - 275s - loss: 127.1589 - val\_loss: 144.8666
Epoch 16/20
101/101 [==============================] - 273s - loss: 124.0815 - val\_loss: 142.5341
Epoch 17/20
101/101 [==============================] - 277s - loss: 120.7425 - val\_loss: 144.8985
Epoch 18/20
101/101 [==============================] - 275s - loss: 117.9136 - val\_loss: 142.1890
Epoch 19/20
101/101 [==============================] - 275s - loss: 115.0634 - val\_loss: 144.0135
Epoch 20/20
101/101 [==============================] - 275s - loss: 112.6410 - val\_loss: 141.2294

    \end{Verbatim}

     \#\#\# (OPTIONAL IMPLEMENTATION) Models 5+

If you would like to try out more architectures than the ones above,
please use the code cell below. Please continue to follow the same
convention for saving the models; for the \(i\)-th sample model, please
save the loss at \textbf{\texttt{model\_i.pickle}} and saving the
trained model at \textbf{\texttt{model\_i.h5}}.

    \begin{Verbatim}[commandchars=\\\{\}]
{\color{incolor}In [{\color{incolor} }]:} \PY{c+c1}{\PYZsh{}\PYZsh{} (Optional) TODO: Try out some more models!}
        \PY{c+c1}{\PYZsh{}\PYZsh{}\PYZsh{} Feel free to use as many code cells as needed.}
\end{Verbatim}


     \#\#\# Compare the Models

Execute the code cell below to evaluate the performance of the drafted
deep learning models. The training and validation loss are plotted for
each model.

    \begin{Verbatim}[commandchars=\\\{\}]
{\color{incolor}In [{\color{incolor}2}]:} \PY{k+kn}{from} \PY{n+nn}{glob} \PY{k}{import} \PY{n}{glob}
        \PY{k+kn}{import} \PY{n+nn}{numpy} \PY{k}{as} \PY{n+nn}{np}
        \PY{k+kn}{import} \PY{n+nn}{\PYZus{}pickle} \PY{k}{as} \PY{n+nn}{pickle}
        \PY{k+kn}{import} \PY{n+nn}{seaborn} \PY{k}{as} \PY{n+nn}{sns}
        \PY{k+kn}{import} \PY{n+nn}{matplotlib}\PY{n+nn}{.}\PY{n+nn}{pyplot} \PY{k}{as} \PY{n+nn}{plt}
        \PY{o}{\PYZpc{}}\PY{k}{matplotlib} inline
        \PY{n}{sns}\PY{o}{.}\PY{n}{set\PYZus{}style}\PY{p}{(}\PY{n}{style}\PY{o}{=}\PY{l+s+s1}{\PYZsq{}}\PY{l+s+s1}{white}\PY{l+s+s1}{\PYZsq{}}\PY{p}{)}
        
        \PY{c+c1}{\PYZsh{} obtain the paths for the saved model history}
        \PY{n}{all\PYZus{}pickles} \PY{o}{=} \PY{n+nb}{sorted}\PY{p}{(}\PY{n}{glob}\PY{p}{(}\PY{l+s+s2}{\PYZdq{}}\PY{l+s+s2}{results/*.pickle}\PY{l+s+s2}{\PYZdq{}}\PY{p}{)}\PY{p}{)}
        \PY{c+c1}{\PYZsh{} extract the name of each model}
        \PY{n}{model\PYZus{}names} \PY{o}{=} \PY{p}{[}\PY{n}{item}\PY{p}{[}\PY{l+m+mi}{8}\PY{p}{:}\PY{o}{\PYZhy{}}\PY{l+m+mi}{7}\PY{p}{]} \PY{k}{for} \PY{n}{item} \PY{o+ow}{in} \PY{n}{all\PYZus{}pickles}\PY{p}{]}
        \PY{c+c1}{\PYZsh{} extract the loss history for each model}
        \PY{n}{valid\PYZus{}loss} \PY{o}{=} \PY{p}{[}\PY{n}{pickle}\PY{o}{.}\PY{n}{load}\PY{p}{(} \PY{n+nb}{open}\PY{p}{(} \PY{n}{i}\PY{p}{,} \PY{l+s+s2}{\PYZdq{}}\PY{l+s+s2}{rb}\PY{l+s+s2}{\PYZdq{}} \PY{p}{)} \PY{p}{)}\PY{p}{[}\PY{l+s+s1}{\PYZsq{}}\PY{l+s+s1}{val\PYZus{}loss}\PY{l+s+s1}{\PYZsq{}}\PY{p}{]} \PY{k}{for} \PY{n}{i} \PY{o+ow}{in} \PY{n}{all\PYZus{}pickles}\PY{p}{]}
        \PY{n}{train\PYZus{}loss} \PY{o}{=} \PY{p}{[}\PY{n}{pickle}\PY{o}{.}\PY{n}{load}\PY{p}{(} \PY{n+nb}{open}\PY{p}{(} \PY{n}{i}\PY{p}{,} \PY{l+s+s2}{\PYZdq{}}\PY{l+s+s2}{rb}\PY{l+s+s2}{\PYZdq{}} \PY{p}{)} \PY{p}{)}\PY{p}{[}\PY{l+s+s1}{\PYZsq{}}\PY{l+s+s1}{loss}\PY{l+s+s1}{\PYZsq{}}\PY{p}{]} \PY{k}{for} \PY{n}{i} \PY{o+ow}{in} \PY{n}{all\PYZus{}pickles}\PY{p}{]}
        \PY{c+c1}{\PYZsh{} save the number of epochs used to train each model}
        \PY{n}{num\PYZus{}epochs} \PY{o}{=} \PY{p}{[}\PY{n+nb}{len}\PY{p}{(}\PY{n}{valid\PYZus{}loss}\PY{p}{[}\PY{n}{i}\PY{p}{]}\PY{p}{)} \PY{k}{for} \PY{n}{i} \PY{o+ow}{in} \PY{n+nb}{range}\PY{p}{(}\PY{n+nb}{len}\PY{p}{(}\PY{n}{valid\PYZus{}loss}\PY{p}{)}\PY{p}{)}\PY{p}{]}
        
        \PY{n}{fig} \PY{o}{=} \PY{n}{plt}\PY{o}{.}\PY{n}{figure}\PY{p}{(}\PY{n}{figsize}\PY{o}{=}\PY{p}{(}\PY{l+m+mi}{16}\PY{p}{,}\PY{l+m+mi}{5}\PY{p}{)}\PY{p}{)}
        
        \PY{c+c1}{\PYZsh{} plot the training loss vs. epoch for each model}
        \PY{n}{ax1} \PY{o}{=} \PY{n}{fig}\PY{o}{.}\PY{n}{add\PYZus{}subplot}\PY{p}{(}\PY{l+m+mi}{121}\PY{p}{)}
        \PY{k}{for} \PY{n}{i} \PY{o+ow}{in} \PY{n+nb}{range}\PY{p}{(}\PY{n+nb}{len}\PY{p}{(}\PY{n}{all\PYZus{}pickles}\PY{p}{)}\PY{p}{)}\PY{p}{:}
            \PY{n}{ax1}\PY{o}{.}\PY{n}{plot}\PY{p}{(}\PY{n}{np}\PY{o}{.}\PY{n}{linspace}\PY{p}{(}\PY{l+m+mi}{1}\PY{p}{,} \PY{n}{num\PYZus{}epochs}\PY{p}{[}\PY{n}{i}\PY{p}{]}\PY{p}{,} \PY{n}{num\PYZus{}epochs}\PY{p}{[}\PY{n}{i}\PY{p}{]}\PY{p}{)}\PY{p}{,} 
                    \PY{n}{train\PYZus{}loss}\PY{p}{[}\PY{n}{i}\PY{p}{]}\PY{p}{,} \PY{n}{label}\PY{o}{=}\PY{n}{model\PYZus{}names}\PY{p}{[}\PY{n}{i}\PY{p}{]}\PY{p}{)}
        \PY{c+c1}{\PYZsh{} clean up the plot}
        \PY{n}{ax1}\PY{o}{.}\PY{n}{legend}\PY{p}{(}\PY{p}{)}  
        \PY{n}{ax1}\PY{o}{.}\PY{n}{set\PYZus{}xlim}\PY{p}{(}\PY{p}{[}\PY{l+m+mi}{1}\PY{p}{,} \PY{n+nb}{max}\PY{p}{(}\PY{n}{num\PYZus{}epochs}\PY{p}{)}\PY{p}{]}\PY{p}{)}
        \PY{n}{plt}\PY{o}{.}\PY{n}{xlabel}\PY{p}{(}\PY{l+s+s1}{\PYZsq{}}\PY{l+s+s1}{Epoch}\PY{l+s+s1}{\PYZsq{}}\PY{p}{)}
        \PY{n}{plt}\PY{o}{.}\PY{n}{ylabel}\PY{p}{(}\PY{l+s+s1}{\PYZsq{}}\PY{l+s+s1}{Training Loss}\PY{l+s+s1}{\PYZsq{}}\PY{p}{)}
        
        \PY{c+c1}{\PYZsh{} plot the validation loss vs. epoch for each model}
        \PY{n}{ax2} \PY{o}{=} \PY{n}{fig}\PY{o}{.}\PY{n}{add\PYZus{}subplot}\PY{p}{(}\PY{l+m+mi}{122}\PY{p}{)}
        \PY{k}{for} \PY{n}{i} \PY{o+ow}{in} \PY{n+nb}{range}\PY{p}{(}\PY{n+nb}{len}\PY{p}{(}\PY{n}{all\PYZus{}pickles}\PY{p}{)}\PY{p}{)}\PY{p}{:}
            \PY{n}{ax2}\PY{o}{.}\PY{n}{plot}\PY{p}{(}\PY{n}{np}\PY{o}{.}\PY{n}{linspace}\PY{p}{(}\PY{l+m+mi}{1}\PY{p}{,} \PY{n}{num\PYZus{}epochs}\PY{p}{[}\PY{n}{i}\PY{p}{]}\PY{p}{,} \PY{n}{num\PYZus{}epochs}\PY{p}{[}\PY{n}{i}\PY{p}{]}\PY{p}{)}\PY{p}{,} 
                    \PY{n}{valid\PYZus{}loss}\PY{p}{[}\PY{n}{i}\PY{p}{]}\PY{p}{,} \PY{n}{label}\PY{o}{=}\PY{n}{model\PYZus{}names}\PY{p}{[}\PY{n}{i}\PY{p}{]}\PY{p}{)}
        \PY{c+c1}{\PYZsh{} clean up the plot}
        \PY{n}{ax2}\PY{o}{.}\PY{n}{legend}\PY{p}{(}\PY{p}{)}  
        \PY{n}{ax2}\PY{o}{.}\PY{n}{set\PYZus{}xlim}\PY{p}{(}\PY{p}{[}\PY{l+m+mi}{1}\PY{p}{,} \PY{n+nb}{max}\PY{p}{(}\PY{n}{num\PYZus{}epochs}\PY{p}{)}\PY{p}{]}\PY{p}{)}
        \PY{n}{plt}\PY{o}{.}\PY{n}{xlabel}\PY{p}{(}\PY{l+s+s1}{\PYZsq{}}\PY{l+s+s1}{Epoch}\PY{l+s+s1}{\PYZsq{}}\PY{p}{)}
        \PY{n}{plt}\PY{o}{.}\PY{n}{ylabel}\PY{p}{(}\PY{l+s+s1}{\PYZsq{}}\PY{l+s+s1}{Validation Loss}\PY{l+s+s1}{\PYZsq{}}\PY{p}{)}
        \PY{n}{plt}\PY{o}{.}\PY{n}{show}\PY{p}{(}\PY{p}{)}
\end{Verbatim}


    \begin{center}
    \adjustimage{max size={0.9\linewidth}{0.9\paperheight}}{output_34_0.png}
    \end{center}
    { \hspace*{\fill} \\}
    
    \textbf{Question 1:} Use the plot above to analyze the performance of
each of the attempted architectures. Which performs best? Provide an
explanation regarding why you think some models perform better than
others.

\textbf{Answer:}

Model 0 preformse the worst becose cannot handle sequence data so is
compleatly unusable for this problem.

Model 1 by introducing TimeDistributed layer to our simple RNN model we
can see some amazing results.

Model 2 by introducing CNN to our model we can now see faster traning
and better feuthers detection but we are starting to overfit.

Model 3 by adding more RNN layers we can see that the model preformse
better as it captures more feuthers but we are also starting to overfit.

Model 4 using Bidirectional RNN seams really usful as it allow us to
have context of the input data it preformse avarage based to the another
models but is not overfiting as much and probably by increasing the
epochs it will ipmrove the results.

     \#\#\# (IMPLEMENTATION) Final Model

Now that you've tried out many sample models, use what you've learned to
draft your own architecture! While your final acoustic model should not
be identical to any of the architectures explored above, you are welcome
to merely combine the explored layers above into a deeper architecture.
It is \textbf{NOT} necessary to include new layer types that were not
explored in the notebook.

However, if you would like some ideas for even more layer types, check
out these ideas for some additional, optional extensions to your model:

\begin{itemize}
\tightlist
\item
  If you notice your model is overfitting to the training dataset,
  consider adding \textbf{dropout}! To add dropout to
  \href{https://faroit.github.io/keras-docs/1.0.2/layers/recurrent/}{recurrent
  layers}, pay special attention to the \texttt{dropout\_W} and
  \texttt{dropout\_U} arguments. This
  \href{http://arxiv.org/abs/1512.05287}{paper} may also provide some
  interesting theoretical background.
\item
  If you choose to include a convolutional layer in your model, you may
  get better results by working with \textbf{dilated convolutions}. If
  you choose to use dilated convolutions, make sure that you are able to
  accurately calculate the length of the acoustic model's output in the
  \texttt{model.output\_length} lambda function. You can read more about
  dilated convolutions in Google's
  \href{https://arxiv.org/abs/1609.03499}{WaveNet paper}. For an example
  of a speech-to-text system that makes use of dilated convolutions,
  check out this GitHub
  \href{https://github.com/buriburisuri/speech-to-text-wavenet}{repository}.
  You can work with dilated convolutions
  \href{https://keras.io/layers/convolutional/}{in Keras} by paying
  special attention to the \texttt{padding} argument when you specify a
  convolutional layer.
\item
  If your model makes use of convolutional layers, why not also
  experiment with adding \textbf{max pooling}? Check out
  \href{https://arxiv.org/pdf/1701.02720.pdf}{this paper} for example
  architecture that makes use of max pooling in an acoustic model.
\item
  So far, you have experimented with a single bidirectional RNN layer.
  Consider stacking the bidirectional layers, to produce a
  \href{https://www.cs.toronto.edu/~graves/asru_2013.pdf}{deep
  bidirectional RNN}!
\end{itemize}

All models that you specify in this repository should have
\texttt{output\_length} defined as an attribute. This attribute is a
lambda function that maps the (temporal) length of the input acoustic
features to the (temporal) length of the output softmax layer. This
function is used in the computation of CTC loss; to see this, look at
the \texttt{add\_ctc\_loss} function in \texttt{train\_utils.py}. To see
where the \texttt{output\_length} attribute is defined for the models in
the code, take a look at the \texttt{sample\_models.py} file. You will
notice this line of code within most models:

\begin{verbatim}
model.output_length = lambda x: x
\end{verbatim}

The acoustic model that incorporates a convolutional layer
(\texttt{cnn\_rnn\_model}) has a line that is a bit different:

\begin{verbatim}
model.output_length = lambda x: cnn_output_length(
        x, kernel_size, conv_border_mode, conv_stride)
\end{verbatim}

In the case of models that use purely recurrent layers, the lambda
function is the identity function, as the recurrent layers do not modify
the (temporal) length of their input tensors. However, convolutional
layers are more complicated and require a specialized function
(\texttt{cnn\_output\_length} in \texttt{sample\_models.py}) to
determine the temporal length of their output.

You will have to add the \texttt{output\_length} attribute to your final
model before running the code cell below. Feel free to use the
\texttt{cnn\_output\_length} function, if it suits your model.

    \begin{Verbatim}[commandchars=\\\{\}]
{\color{incolor}In [{\color{incolor}40}]:} \PY{c+c1}{\PYZsh{} specify the model}
         \PY{n}{model\PYZus{}end} \PY{o}{=} \PY{n}{final\PYZus{}model}\PY{p}{(}\PY{n}{input\PYZus{}dim}\PY{o}{=}\PY{l+m+mi}{161}\PY{p}{,} \PY{c+c1}{\PYZsh{} change to 13 if you would like to use MFCC features}
                                 \PY{n}{filters}\PY{o}{=}\PY{l+m+mi}{200}\PY{p}{,}
                                 \PY{n}{kernel\PYZus{}size}\PY{o}{=}\PY{l+m+mi}{11}\PY{p}{,} 
                                 \PY{n}{conv\PYZus{}stride}\PY{o}{=}\PY{l+m+mi}{2}\PY{p}{,}
                                 \PY{n}{conv\PYZus{}border\PYZus{}mode}\PY{o}{=}\PY{l+s+s1}{\PYZsq{}}\PY{l+s+s1}{valid}\PY{l+s+s1}{\PYZsq{}}\PY{p}{,}
                                 \PY{n}{units}\PY{o}{=}\PY{l+m+mi}{200}\PY{p}{,}
                                 \PY{n}{recur\PYZus{}layers}\PY{o}{=}\PY{l+m+mi}{4}\PY{p}{,}
                                 \PY{n}{drop\PYZus{}out}\PY{o}{=}\PY{l+m+mf}{0.2}\PY{p}{)}
\end{Verbatim}


    \begin{Verbatim}[commandchars=\\\{\}]
\_\_\_\_\_\_\_\_\_\_\_\_\_\_\_\_\_\_\_\_\_\_\_\_\_\_\_\_\_\_\_\_\_\_\_\_\_\_\_\_\_\_\_\_\_\_\_\_\_\_\_\_\_\_\_\_\_\_\_\_\_\_\_\_\_
Layer (type)                 Output Shape              Param \#   
=================================================================
the\_input (InputLayer)       (None, None, 161)         0         
\_\_\_\_\_\_\_\_\_\_\_\_\_\_\_\_\_\_\_\_\_\_\_\_\_\_\_\_\_\_\_\_\_\_\_\_\_\_\_\_\_\_\_\_\_\_\_\_\_\_\_\_\_\_\_\_\_\_\_\_\_\_\_\_\_
conv1d (Conv1D)              (None, None, 200)         354400    
\_\_\_\_\_\_\_\_\_\_\_\_\_\_\_\_\_\_\_\_\_\_\_\_\_\_\_\_\_\_\_\_\_\_\_\_\_\_\_\_\_\_\_\_\_\_\_\_\_\_\_\_\_\_\_\_\_\_\_\_\_\_\_\_\_
batch\_normalization\_66 (Batc (None, None, 200)         800       
\_\_\_\_\_\_\_\_\_\_\_\_\_\_\_\_\_\_\_\_\_\_\_\_\_\_\_\_\_\_\_\_\_\_\_\_\_\_\_\_\_\_\_\_\_\_\_\_\_\_\_\_\_\_\_\_\_\_\_\_\_\_\_\_\_
gru\_31 (GRU)                 (None, None, 200)         240600    
\_\_\_\_\_\_\_\_\_\_\_\_\_\_\_\_\_\_\_\_\_\_\_\_\_\_\_\_\_\_\_\_\_\_\_\_\_\_\_\_\_\_\_\_\_\_\_\_\_\_\_\_\_\_\_\_\_\_\_\_\_\_\_\_\_
batch\_normalization\_67 (Batc (None, None, 200)         800       
\_\_\_\_\_\_\_\_\_\_\_\_\_\_\_\_\_\_\_\_\_\_\_\_\_\_\_\_\_\_\_\_\_\_\_\_\_\_\_\_\_\_\_\_\_\_\_\_\_\_\_\_\_\_\_\_\_\_\_\_\_\_\_\_\_
gru\_32 (GRU)                 (None, None, 200)         240600    
\_\_\_\_\_\_\_\_\_\_\_\_\_\_\_\_\_\_\_\_\_\_\_\_\_\_\_\_\_\_\_\_\_\_\_\_\_\_\_\_\_\_\_\_\_\_\_\_\_\_\_\_\_\_\_\_\_\_\_\_\_\_\_\_\_
batch\_normalization\_68 (Batc (None, None, 200)         800       
\_\_\_\_\_\_\_\_\_\_\_\_\_\_\_\_\_\_\_\_\_\_\_\_\_\_\_\_\_\_\_\_\_\_\_\_\_\_\_\_\_\_\_\_\_\_\_\_\_\_\_\_\_\_\_\_\_\_\_\_\_\_\_\_\_
gru\_33 (GRU)                 (None, None, 200)         240600    
\_\_\_\_\_\_\_\_\_\_\_\_\_\_\_\_\_\_\_\_\_\_\_\_\_\_\_\_\_\_\_\_\_\_\_\_\_\_\_\_\_\_\_\_\_\_\_\_\_\_\_\_\_\_\_\_\_\_\_\_\_\_\_\_\_
batch\_normalization\_69 (Batc (None, None, 200)         800       
\_\_\_\_\_\_\_\_\_\_\_\_\_\_\_\_\_\_\_\_\_\_\_\_\_\_\_\_\_\_\_\_\_\_\_\_\_\_\_\_\_\_\_\_\_\_\_\_\_\_\_\_\_\_\_\_\_\_\_\_\_\_\_\_\_
gru\_34 (GRU)                 (None, None, 200)         240600    
\_\_\_\_\_\_\_\_\_\_\_\_\_\_\_\_\_\_\_\_\_\_\_\_\_\_\_\_\_\_\_\_\_\_\_\_\_\_\_\_\_\_\_\_\_\_\_\_\_\_\_\_\_\_\_\_\_\_\_\_\_\_\_\_\_
batch\_normalization\_70 (Batc (None, None, 200)         800       
\_\_\_\_\_\_\_\_\_\_\_\_\_\_\_\_\_\_\_\_\_\_\_\_\_\_\_\_\_\_\_\_\_\_\_\_\_\_\_\_\_\_\_\_\_\_\_\_\_\_\_\_\_\_\_\_\_\_\_\_\_\_\_\_\_
time\_distributed\_20 (TimeDis (None, None, 29)          5829      
\_\_\_\_\_\_\_\_\_\_\_\_\_\_\_\_\_\_\_\_\_\_\_\_\_\_\_\_\_\_\_\_\_\_\_\_\_\_\_\_\_\_\_\_\_\_\_\_\_\_\_\_\_\_\_\_\_\_\_\_\_\_\_\_\_
softmax (Activation)         (None, None, 29)          0         
=================================================================
Total params: 1,326,629
Trainable params: 1,324,629
Non-trainable params: 2,000
\_\_\_\_\_\_\_\_\_\_\_\_\_\_\_\_\_\_\_\_\_\_\_\_\_\_\_\_\_\_\_\_\_\_\_\_\_\_\_\_\_\_\_\_\_\_\_\_\_\_\_\_\_\_\_\_\_\_\_\_\_\_\_\_\_
None

    \end{Verbatim}

    Please execute the code cell below to train the neural network you
specified in \texttt{input\_to\_softmax}. After the model has finished
training, the model is
\href{https://keras.io/getting-started/faq/\#how-can-i-save-a-keras-model}{saved}
in the HDF5 file \texttt{model\_end.h5}. The loss history is
\href{https://wiki.python.org/moin/UsingPickle}{saved} in
\texttt{model\_end.pickle}. You are welcome to tweak any of the optional
parameters while calling the \texttt{train\_model} function, but this is
not required.

    \begin{Verbatim}[commandchars=\\\{\}]
{\color{incolor}In [{\color{incolor}41}]:} \PY{n}{train\PYZus{}model}\PY{p}{(}\PY{n}{input\PYZus{}to\PYZus{}softmax}\PY{o}{=}\PY{n}{model\PYZus{}end}\PY{p}{,} 
                     \PY{n}{pickle\PYZus{}path}\PY{o}{=}\PY{l+s+s1}{\PYZsq{}}\PY{l+s+s1}{model\PYZus{}end.pickle}\PY{l+s+s1}{\PYZsq{}}\PY{p}{,} 
                     \PY{n}{save\PYZus{}model\PYZus{}path}\PY{o}{=}\PY{l+s+s1}{\PYZsq{}}\PY{l+s+s1}{model\PYZus{}end.h5}\PY{l+s+s1}{\PYZsq{}}\PY{p}{,} 
                     \PY{n}{spectrogram}\PY{o}{=}\PY{k+kc}{True}\PY{p}{,} \PY{c+c1}{\PYZsh{} change to False if you would like to use MFCC features}
                     \PY{n}{epochs}\PY{o}{=}\PY{l+m+mi}{40}\PY{p}{)} 
\end{Verbatim}


    \begin{Verbatim}[commandchars=\\\{\}]
Epoch 1/40
101/101 [==============================] - 286s - loss: 269.2321 - val\_loss: 233.5749
Epoch 2/40
101/101 [==============================] - 281s - loss: 213.2487 - val\_loss: 191.3667
Epoch 3/40
101/101 [==============================] - 272s - loss: 182.8210 - val\_loss: 162.5092
Epoch 4/40
101/101 [==============================] - 281s - loss: 165.0535 - val\_loss: 153.4251
Epoch 5/40
101/101 [==============================] - 275s - loss: 153.4889 - val\_loss: 142.1481
Epoch 6/40
101/101 [==============================] - 275s - loss: 144.2067 - val\_loss: 140.3627
Epoch 7/40
101/101 [==============================] - 273s - loss: 137.3124 - val\_loss: 132.1809
Epoch 8/40
101/101 [==============================] - 272s - loss: 131.6903 - val\_loss: 127.4366
Epoch 9/40
101/101 [==============================] - 273s - loss: 127.1162 - val\_loss: 125.7950
Epoch 10/40
101/101 [==============================] - 272s - loss: 122.5956 - val\_loss: 120.5612
Epoch 11/40
101/101 [==============================] - 272s - loss: 119.1977 - val\_loss: 119.1196
Epoch 12/40
101/101 [==============================] - 273s - loss: 116.1248 - val\_loss: 119.4280
Epoch 13/40
101/101 [==============================] - 272s - loss: 113.2969 - val\_loss: 116.1005
Epoch 14/40
101/101 [==============================] - 271s - loss: 110.4095 - val\_loss: 114.3789
Epoch 15/40
101/101 [==============================] - 272s - loss: 108.4032 - val\_loss: 115.2235
Epoch 16/40
101/101 [==============================] - 273s - loss: 105.0037 - val\_loss: 113.8735
Epoch 17/40
101/101 [==============================] - 275s - loss: 103.0334 - val\_loss: 112.9639
Epoch 18/40
101/101 [==============================] - 295s - loss: 101.2784 - val\_loss: 112.5247
Epoch 19/40
101/101 [==============================] - 298s - loss: 98.7319 - val\_loss: 110.1100
Epoch 20/40
101/101 [==============================] - 296s - loss: 97.1200 - val\_loss: 110.1599
Epoch 21/40
101/101 [==============================] - 298s - loss: 95.3172 - val\_loss: 106.7017
Epoch 22/40
101/101 [==============================] - 299s - loss: 94.1226 - val\_loss: 109.6103
Epoch 23/40
101/101 [==============================] - 297s - loss: 92.3061 - val\_loss: 108.3446
Epoch 24/40
101/101 [==============================] - 297s - loss: 90.3997 - val\_loss: 107.7179
Epoch 25/40
101/101 [==============================] - 298s - loss: 89.6657 - val\_loss: 108.2435
Epoch 26/40
101/101 [==============================] - 297s - loss: 88.0548 - val\_loss: 111.3113
Epoch 27/40
101/101 [==============================] - 295s - loss: 86.5643 - val\_loss: 106.4508
Epoch 28/40
101/101 [==============================] - 294s - loss: 85.2720 - val\_loss: 105.3537
Epoch 29/40
101/101 [==============================] - 297s - loss: 84.1716 - val\_loss: 105.8501
Epoch 30/40
101/101 [==============================] - 294s - loss: 82.6955 - val\_loss: 104.6581
Epoch 31/40
101/101 [==============================] - 296s - loss: 82.0304 - val\_loss: 104.2280
Epoch 32/40
101/101 [==============================] - 296s - loss: 81.1303 - val\_loss: 104.6346
Epoch 33/40
101/101 [==============================] - 294s - loss: 79.3556 - val\_loss: 104.0187
Epoch 34/40
101/101 [==============================] - 296s - loss: 79.1561 - val\_loss: 106.3322
Epoch 35/40
101/101 [==============================] - 295s - loss: 77.8495 - val\_loss: 104.4979
Epoch 36/40
101/101 [==============================] - 296s - loss: 76.6495 - val\_loss: 104.8944
Epoch 37/40
101/101 [==============================] - 294s - loss: 76.0853 - val\_loss: 103.7641
Epoch 38/40
101/101 [==============================] - 294s - loss: 74.9576 - val\_loss: 104.3076
Epoch 39/40
101/101 [==============================] - 293s - loss: 74.2689 - val\_loss: 105.5964
Epoch 40/40
101/101 [==============================] - 296s - loss: 73.7029 - val\_loss: 104.6491

    \end{Verbatim}

    \textbf{Question 2:} Describe your final model architecture and your
reasoning at each step.

\textbf{Answer:}

I wanted to use Bidirectional model but becose of the training time
constrain I end up using CNN + deep RNN + TimeDistributed Dense model I
also added some drop out to make sure i limit the overfiting.

     \#\# STEP 3: Obtain Predictions

We have written a function for you to decode the predictions of your
acoustic model. To use the function, please execute the code cell below.

    \begin{Verbatim}[commandchars=\\\{\}]
{\color{incolor}In [{\color{incolor}42}]:} \PY{k+kn}{import} \PY{n+nn}{numpy} \PY{k}{as} \PY{n+nn}{np}
         \PY{k+kn}{from} \PY{n+nn}{data\PYZus{}generator} \PY{k}{import} \PY{n}{AudioGenerator}
         \PY{k+kn}{from} \PY{n+nn}{keras} \PY{k}{import} \PY{n}{backend} \PY{k}{as} \PY{n}{K}
         \PY{k+kn}{from} \PY{n+nn}{utils} \PY{k}{import} \PY{n}{int\PYZus{}sequence\PYZus{}to\PYZus{}text}
         \PY{k+kn}{from} \PY{n+nn}{IPython}\PY{n+nn}{.}\PY{n+nn}{display} \PY{k}{import} \PY{n}{Audio}
         
         \PY{k}{def} \PY{n+nf}{get\PYZus{}predictions}\PY{p}{(}\PY{n}{index}\PY{p}{,} \PY{n}{partition}\PY{p}{,} \PY{n}{input\PYZus{}to\PYZus{}softmax}\PY{p}{,} \PY{n}{model\PYZus{}path}\PY{p}{)}\PY{p}{:}
             \PY{l+s+sd}{\PYZdq{}\PYZdq{}\PYZdq{} Print a model\PYZsq{}s decoded predictions}
         \PY{l+s+sd}{    Params:}
         \PY{l+s+sd}{        index (int): The example you would like to visualize}
         \PY{l+s+sd}{        partition (str): One of \PYZsq{}train\PYZsq{} or \PYZsq{}validation\PYZsq{}}
         \PY{l+s+sd}{        input\PYZus{}to\PYZus{}softmax (Model): The acoustic model}
         \PY{l+s+sd}{        model\PYZus{}path (str): Path to saved acoustic model\PYZsq{}s weights}
         \PY{l+s+sd}{    \PYZdq{}\PYZdq{}\PYZdq{}}
             \PY{c+c1}{\PYZsh{} load the train and test data}
             \PY{n}{data\PYZus{}gen} \PY{o}{=} \PY{n}{AudioGenerator}\PY{p}{(}\PY{p}{)}
             \PY{n}{data\PYZus{}gen}\PY{o}{.}\PY{n}{load\PYZus{}train\PYZus{}data}\PY{p}{(}\PY{p}{)}
             \PY{n}{data\PYZus{}gen}\PY{o}{.}\PY{n}{load\PYZus{}validation\PYZus{}data}\PY{p}{(}\PY{p}{)}
             
             \PY{c+c1}{\PYZsh{} obtain the true transcription and the audio features }
             \PY{k}{if} \PY{n}{partition} \PY{o}{==} \PY{l+s+s1}{\PYZsq{}}\PY{l+s+s1}{validation}\PY{l+s+s1}{\PYZsq{}}\PY{p}{:}
                 \PY{n}{transcr} \PY{o}{=} \PY{n}{data\PYZus{}gen}\PY{o}{.}\PY{n}{valid\PYZus{}texts}\PY{p}{[}\PY{n}{index}\PY{p}{]}
                 \PY{n}{audio\PYZus{}path} \PY{o}{=} \PY{n}{data\PYZus{}gen}\PY{o}{.}\PY{n}{valid\PYZus{}audio\PYZus{}paths}\PY{p}{[}\PY{n}{index}\PY{p}{]}
                 \PY{n}{data\PYZus{}point} \PY{o}{=} \PY{n}{data\PYZus{}gen}\PY{o}{.}\PY{n}{normalize}\PY{p}{(}\PY{n}{data\PYZus{}gen}\PY{o}{.}\PY{n}{featurize}\PY{p}{(}\PY{n}{audio\PYZus{}path}\PY{p}{)}\PY{p}{)}
             \PY{k}{elif} \PY{n}{partition} \PY{o}{==} \PY{l+s+s1}{\PYZsq{}}\PY{l+s+s1}{train}\PY{l+s+s1}{\PYZsq{}}\PY{p}{:}
                 \PY{n}{transcr} \PY{o}{=} \PY{n}{data\PYZus{}gen}\PY{o}{.}\PY{n}{train\PYZus{}texts}\PY{p}{[}\PY{n}{index}\PY{p}{]}
                 \PY{n}{audio\PYZus{}path} \PY{o}{=} \PY{n}{data\PYZus{}gen}\PY{o}{.}\PY{n}{train\PYZus{}audio\PYZus{}paths}\PY{p}{[}\PY{n}{index}\PY{p}{]}
                 \PY{n}{data\PYZus{}point} \PY{o}{=} \PY{n}{data\PYZus{}gen}\PY{o}{.}\PY{n}{normalize}\PY{p}{(}\PY{n}{data\PYZus{}gen}\PY{o}{.}\PY{n}{featurize}\PY{p}{(}\PY{n}{audio\PYZus{}path}\PY{p}{)}\PY{p}{)}
             \PY{k}{else}\PY{p}{:}
                 \PY{k}{raise} \PY{n+ne}{Exception}\PY{p}{(}\PY{l+s+s1}{\PYZsq{}}\PY{l+s+s1}{Invalid partition!  Must be }\PY{l+s+s1}{\PYZdq{}}\PY{l+s+s1}{train}\PY{l+s+s1}{\PYZdq{}}\PY{l+s+s1}{ or }\PY{l+s+s1}{\PYZdq{}}\PY{l+s+s1}{validation}\PY{l+s+s1}{\PYZdq{}}\PY{l+s+s1}{\PYZsq{}}\PY{p}{)}
                 
             \PY{c+c1}{\PYZsh{} obtain and decode the acoustic model\PYZsq{}s predictions}
             \PY{n}{input\PYZus{}to\PYZus{}softmax}\PY{o}{.}\PY{n}{load\PYZus{}weights}\PY{p}{(}\PY{n}{model\PYZus{}path}\PY{p}{)}
             \PY{n}{prediction} \PY{o}{=} \PY{n}{input\PYZus{}to\PYZus{}softmax}\PY{o}{.}\PY{n}{predict}\PY{p}{(}\PY{n}{np}\PY{o}{.}\PY{n}{expand\PYZus{}dims}\PY{p}{(}\PY{n}{data\PYZus{}point}\PY{p}{,} \PY{n}{axis}\PY{o}{=}\PY{l+m+mi}{0}\PY{p}{)}\PY{p}{)}
             \PY{n}{output\PYZus{}length} \PY{o}{=} \PY{p}{[}\PY{n}{input\PYZus{}to\PYZus{}softmax}\PY{o}{.}\PY{n}{output\PYZus{}length}\PY{p}{(}\PY{n}{data\PYZus{}point}\PY{o}{.}\PY{n}{shape}\PY{p}{[}\PY{l+m+mi}{0}\PY{p}{]}\PY{p}{)}\PY{p}{]} 
             \PY{n}{pred\PYZus{}ints} \PY{o}{=} \PY{p}{(}\PY{n}{K}\PY{o}{.}\PY{n}{eval}\PY{p}{(}\PY{n}{K}\PY{o}{.}\PY{n}{ctc\PYZus{}decode}\PY{p}{(}
                         \PY{n}{prediction}\PY{p}{,} \PY{n}{output\PYZus{}length}\PY{p}{)}\PY{p}{[}\PY{l+m+mi}{0}\PY{p}{]}\PY{p}{[}\PY{l+m+mi}{0}\PY{p}{]}\PY{p}{)}\PY{o}{+}\PY{l+m+mi}{1}\PY{p}{)}\PY{o}{.}\PY{n}{flatten}\PY{p}{(}\PY{p}{)}\PY{o}{.}\PY{n}{tolist}\PY{p}{(}\PY{p}{)}
             
             \PY{c+c1}{\PYZsh{} play the audio file, and display the true and predicted transcriptions}
             \PY{n+nb}{print}\PY{p}{(}\PY{l+s+s1}{\PYZsq{}}\PY{l+s+s1}{\PYZhy{}}\PY{l+s+s1}{\PYZsq{}}\PY{o}{*}\PY{l+m+mi}{80}\PY{p}{)}
             \PY{n}{Audio}\PY{p}{(}\PY{n}{audio\PYZus{}path}\PY{p}{)}
             \PY{n+nb}{print}\PY{p}{(}\PY{l+s+s1}{\PYZsq{}}\PY{l+s+s1}{True transcription:}\PY{l+s+se}{\PYZbs{}n}\PY{l+s+s1}{\PYZsq{}} \PY{o}{+} \PY{l+s+s1}{\PYZsq{}}\PY{l+s+se}{\PYZbs{}n}\PY{l+s+s1}{\PYZsq{}} \PY{o}{+} \PY{n}{transcr}\PY{p}{)}
             \PY{n+nb}{print}\PY{p}{(}\PY{l+s+s1}{\PYZsq{}}\PY{l+s+s1}{\PYZhy{}}\PY{l+s+s1}{\PYZsq{}}\PY{o}{*}\PY{l+m+mi}{80}\PY{p}{)}
             \PY{n+nb}{print}\PY{p}{(}\PY{l+s+s1}{\PYZsq{}}\PY{l+s+s1}{Predicted transcription:}\PY{l+s+se}{\PYZbs{}n}\PY{l+s+s1}{\PYZsq{}} \PY{o}{+} \PY{l+s+s1}{\PYZsq{}}\PY{l+s+se}{\PYZbs{}n}\PY{l+s+s1}{\PYZsq{}} \PY{o}{+} \PY{l+s+s1}{\PYZsq{}}\PY{l+s+s1}{\PYZsq{}}\PY{o}{.}\PY{n}{join}\PY{p}{(}\PY{n}{int\PYZus{}sequence\PYZus{}to\PYZus{}text}\PY{p}{(}\PY{n}{pred\PYZus{}ints}\PY{p}{)}\PY{p}{)}\PY{p}{)}
             \PY{n+nb}{print}\PY{p}{(}\PY{l+s+s1}{\PYZsq{}}\PY{l+s+s1}{\PYZhy{}}\PY{l+s+s1}{\PYZsq{}}\PY{o}{*}\PY{l+m+mi}{80}\PY{p}{)}
\end{Verbatim}


    Use the code cell below to obtain the transcription predicted by your
final model for the first example in the training dataset.

    \begin{Verbatim}[commandchars=\\\{\}]
{\color{incolor}In [{\color{incolor}44}]:} \PY{n}{get\PYZus{}predictions}\PY{p}{(}\PY{n}{index}\PY{o}{=}\PY{l+m+mi}{0}\PY{p}{,} 
                         \PY{n}{partition}\PY{o}{=}\PY{l+s+s1}{\PYZsq{}}\PY{l+s+s1}{train}\PY{l+s+s1}{\PYZsq{}}\PY{p}{,}
                         \PY{n}{input\PYZus{}to\PYZus{}softmax}\PY{o}{=}\PY{n}{final\PYZus{}model}\PY{p}{(}
                                 \PY{n}{input\PYZus{}dim}\PY{o}{=}\PY{l+m+mi}{161}\PY{p}{,} \PY{c+c1}{\PYZsh{} change to 13 if you would like to use MFCC features}
                                 \PY{n}{filters}\PY{o}{=}\PY{l+m+mi}{200}\PY{p}{,}
                                 \PY{n}{kernel\PYZus{}size}\PY{o}{=}\PY{l+m+mi}{11}\PY{p}{,} 
                                 \PY{n}{conv\PYZus{}stride}\PY{o}{=}\PY{l+m+mi}{2}\PY{p}{,}
                                 \PY{n}{conv\PYZus{}border\PYZus{}mode}\PY{o}{=}\PY{l+s+s1}{\PYZsq{}}\PY{l+s+s1}{valid}\PY{l+s+s1}{\PYZsq{}}\PY{p}{,}
                                 \PY{n}{units}\PY{o}{=}\PY{l+m+mi}{200}\PY{p}{,}
                                 \PY{n}{recur\PYZus{}layers}\PY{o}{=}\PY{l+m+mi}{4}\PY{p}{,}
                                 \PY{n}{drop\PYZus{}out}\PY{o}{=}\PY{l+m+mf}{0.2}\PY{p}{)}\PY{p}{,} 
                         \PY{n}{model\PYZus{}path}\PY{o}{=}\PY{l+s+s1}{\PYZsq{}}\PY{l+s+s1}{results}\PY{l+s+s1}{\PYZbs{}}\PY{l+s+s1}{model\PYZus{}end.h5}\PY{l+s+s1}{\PYZsq{}}\PY{p}{)}
\end{Verbatim}


    \begin{Verbatim}[commandchars=\\\{\}]
\_\_\_\_\_\_\_\_\_\_\_\_\_\_\_\_\_\_\_\_\_\_\_\_\_\_\_\_\_\_\_\_\_\_\_\_\_\_\_\_\_\_\_\_\_\_\_\_\_\_\_\_\_\_\_\_\_\_\_\_\_\_\_\_\_
Layer (type)                 Output Shape              Param \#   
=================================================================
the\_input (InputLayer)       (None, None, 161)         0         
\_\_\_\_\_\_\_\_\_\_\_\_\_\_\_\_\_\_\_\_\_\_\_\_\_\_\_\_\_\_\_\_\_\_\_\_\_\_\_\_\_\_\_\_\_\_\_\_\_\_\_\_\_\_\_\_\_\_\_\_\_\_\_\_\_
conv1d (Conv1D)              (None, None, 200)         354400    
\_\_\_\_\_\_\_\_\_\_\_\_\_\_\_\_\_\_\_\_\_\_\_\_\_\_\_\_\_\_\_\_\_\_\_\_\_\_\_\_\_\_\_\_\_\_\_\_\_\_\_\_\_\_\_\_\_\_\_\_\_\_\_\_\_
batch\_normalization\_75 (Batc (None, None, 200)         800       
\_\_\_\_\_\_\_\_\_\_\_\_\_\_\_\_\_\_\_\_\_\_\_\_\_\_\_\_\_\_\_\_\_\_\_\_\_\_\_\_\_\_\_\_\_\_\_\_\_\_\_\_\_\_\_\_\_\_\_\_\_\_\_\_\_
gru\_38 (GRU)                 (None, None, 200)         240600    
\_\_\_\_\_\_\_\_\_\_\_\_\_\_\_\_\_\_\_\_\_\_\_\_\_\_\_\_\_\_\_\_\_\_\_\_\_\_\_\_\_\_\_\_\_\_\_\_\_\_\_\_\_\_\_\_\_\_\_\_\_\_\_\_\_
batch\_normalization\_76 (Batc (None, None, 200)         800       
\_\_\_\_\_\_\_\_\_\_\_\_\_\_\_\_\_\_\_\_\_\_\_\_\_\_\_\_\_\_\_\_\_\_\_\_\_\_\_\_\_\_\_\_\_\_\_\_\_\_\_\_\_\_\_\_\_\_\_\_\_\_\_\_\_
gru\_39 (GRU)                 (None, None, 200)         240600    
\_\_\_\_\_\_\_\_\_\_\_\_\_\_\_\_\_\_\_\_\_\_\_\_\_\_\_\_\_\_\_\_\_\_\_\_\_\_\_\_\_\_\_\_\_\_\_\_\_\_\_\_\_\_\_\_\_\_\_\_\_\_\_\_\_
batch\_normalization\_77 (Batc (None, None, 200)         800       
\_\_\_\_\_\_\_\_\_\_\_\_\_\_\_\_\_\_\_\_\_\_\_\_\_\_\_\_\_\_\_\_\_\_\_\_\_\_\_\_\_\_\_\_\_\_\_\_\_\_\_\_\_\_\_\_\_\_\_\_\_\_\_\_\_
gru\_40 (GRU)                 (None, None, 200)         240600    
\_\_\_\_\_\_\_\_\_\_\_\_\_\_\_\_\_\_\_\_\_\_\_\_\_\_\_\_\_\_\_\_\_\_\_\_\_\_\_\_\_\_\_\_\_\_\_\_\_\_\_\_\_\_\_\_\_\_\_\_\_\_\_\_\_
batch\_normalization\_78 (Batc (None, None, 200)         800       
\_\_\_\_\_\_\_\_\_\_\_\_\_\_\_\_\_\_\_\_\_\_\_\_\_\_\_\_\_\_\_\_\_\_\_\_\_\_\_\_\_\_\_\_\_\_\_\_\_\_\_\_\_\_\_\_\_\_\_\_\_\_\_\_\_
gru\_41 (GRU)                 (None, None, 200)         240600    
\_\_\_\_\_\_\_\_\_\_\_\_\_\_\_\_\_\_\_\_\_\_\_\_\_\_\_\_\_\_\_\_\_\_\_\_\_\_\_\_\_\_\_\_\_\_\_\_\_\_\_\_\_\_\_\_\_\_\_\_\_\_\_\_\_
batch\_normalization\_79 (Batc (None, None, 200)         800       
\_\_\_\_\_\_\_\_\_\_\_\_\_\_\_\_\_\_\_\_\_\_\_\_\_\_\_\_\_\_\_\_\_\_\_\_\_\_\_\_\_\_\_\_\_\_\_\_\_\_\_\_\_\_\_\_\_\_\_\_\_\_\_\_\_
time\_distributed\_22 (TimeDis (None, None, 29)          5829      
\_\_\_\_\_\_\_\_\_\_\_\_\_\_\_\_\_\_\_\_\_\_\_\_\_\_\_\_\_\_\_\_\_\_\_\_\_\_\_\_\_\_\_\_\_\_\_\_\_\_\_\_\_\_\_\_\_\_\_\_\_\_\_\_\_
softmax (Activation)         (None, None, 29)          0         
=================================================================
Total params: 1,326,629
Trainable params: 1,324,629
Non-trainable params: 2,000
\_\_\_\_\_\_\_\_\_\_\_\_\_\_\_\_\_\_\_\_\_\_\_\_\_\_\_\_\_\_\_\_\_\_\_\_\_\_\_\_\_\_\_\_\_\_\_\_\_\_\_\_\_\_\_\_\_\_\_\_\_\_\_\_\_
None
--------------------------------------------------------------------------------
True transcription:

mister quilter is the apostle of the middle classes and we are glad to welcome his gospel
--------------------------------------------------------------------------------
Predicted transcription:

mis ter crillter is the apposle of the mittle classes and weare glad towelcom his gospe
--------------------------------------------------------------------------------

    \end{Verbatim}

    Use the next code cell to visualize the model's prediction for the first
example in the validation dataset.

    \begin{Verbatim}[commandchars=\\\{\}]
{\color{incolor}In [{\color{incolor}45}]:} \PY{n}{get\PYZus{}predictions}\PY{p}{(}\PY{n}{index}\PY{o}{=}\PY{l+m+mi}{0}\PY{p}{,} 
                         \PY{n}{partition}\PY{o}{=}\PY{l+s+s1}{\PYZsq{}}\PY{l+s+s1}{validation}\PY{l+s+s1}{\PYZsq{}}\PY{p}{,}
                         \PY{n}{input\PYZus{}to\PYZus{}softmax}\PY{o}{=}\PY{n}{final\PYZus{}model}\PY{p}{(}
                                 \PY{n}{input\PYZus{}dim}\PY{o}{=}\PY{l+m+mi}{161}\PY{p}{,} \PY{c+c1}{\PYZsh{} change to 13 if you would like to use MFCC features}
                                 \PY{n}{filters}\PY{o}{=}\PY{l+m+mi}{200}\PY{p}{,}
                                 \PY{n}{kernel\PYZus{}size}\PY{o}{=}\PY{l+m+mi}{11}\PY{p}{,} 
                                 \PY{n}{conv\PYZus{}stride}\PY{o}{=}\PY{l+m+mi}{2}\PY{p}{,}
                                 \PY{n}{conv\PYZus{}border\PYZus{}mode}\PY{o}{=}\PY{l+s+s1}{\PYZsq{}}\PY{l+s+s1}{valid}\PY{l+s+s1}{\PYZsq{}}\PY{p}{,}
                                 \PY{n}{units}\PY{o}{=}\PY{l+m+mi}{200}\PY{p}{,}
                                 \PY{n}{recur\PYZus{}layers}\PY{o}{=}\PY{l+m+mi}{4}\PY{p}{,}
                                 \PY{n}{drop\PYZus{}out}\PY{o}{=}\PY{l+m+mf}{0.2}\PY{p}{)}\PY{p}{,} 
                         \PY{n}{model\PYZus{}path}\PY{o}{=}\PY{l+s+s1}{\PYZsq{}}\PY{l+s+s1}{results}\PY{l+s+s1}{\PYZbs{}}\PY{l+s+s1}{model\PYZus{}end.h5}\PY{l+s+s1}{\PYZsq{}}\PY{p}{)}
\end{Verbatim}


    \begin{Verbatim}[commandchars=\\\{\}]
\_\_\_\_\_\_\_\_\_\_\_\_\_\_\_\_\_\_\_\_\_\_\_\_\_\_\_\_\_\_\_\_\_\_\_\_\_\_\_\_\_\_\_\_\_\_\_\_\_\_\_\_\_\_\_\_\_\_\_\_\_\_\_\_\_
Layer (type)                 Output Shape              Param \#   
=================================================================
the\_input (InputLayer)       (None, None, 161)         0         
\_\_\_\_\_\_\_\_\_\_\_\_\_\_\_\_\_\_\_\_\_\_\_\_\_\_\_\_\_\_\_\_\_\_\_\_\_\_\_\_\_\_\_\_\_\_\_\_\_\_\_\_\_\_\_\_\_\_\_\_\_\_\_\_\_
conv1d (Conv1D)              (None, None, 200)         354400    
\_\_\_\_\_\_\_\_\_\_\_\_\_\_\_\_\_\_\_\_\_\_\_\_\_\_\_\_\_\_\_\_\_\_\_\_\_\_\_\_\_\_\_\_\_\_\_\_\_\_\_\_\_\_\_\_\_\_\_\_\_\_\_\_\_
batch\_normalization\_80 (Batc (None, None, 200)         800       
\_\_\_\_\_\_\_\_\_\_\_\_\_\_\_\_\_\_\_\_\_\_\_\_\_\_\_\_\_\_\_\_\_\_\_\_\_\_\_\_\_\_\_\_\_\_\_\_\_\_\_\_\_\_\_\_\_\_\_\_\_\_\_\_\_
gru\_42 (GRU)                 (None, None, 200)         240600    
\_\_\_\_\_\_\_\_\_\_\_\_\_\_\_\_\_\_\_\_\_\_\_\_\_\_\_\_\_\_\_\_\_\_\_\_\_\_\_\_\_\_\_\_\_\_\_\_\_\_\_\_\_\_\_\_\_\_\_\_\_\_\_\_\_
batch\_normalization\_81 (Batc (None, None, 200)         800       
\_\_\_\_\_\_\_\_\_\_\_\_\_\_\_\_\_\_\_\_\_\_\_\_\_\_\_\_\_\_\_\_\_\_\_\_\_\_\_\_\_\_\_\_\_\_\_\_\_\_\_\_\_\_\_\_\_\_\_\_\_\_\_\_\_
gru\_43 (GRU)                 (None, None, 200)         240600    
\_\_\_\_\_\_\_\_\_\_\_\_\_\_\_\_\_\_\_\_\_\_\_\_\_\_\_\_\_\_\_\_\_\_\_\_\_\_\_\_\_\_\_\_\_\_\_\_\_\_\_\_\_\_\_\_\_\_\_\_\_\_\_\_\_
batch\_normalization\_82 (Batc (None, None, 200)         800       
\_\_\_\_\_\_\_\_\_\_\_\_\_\_\_\_\_\_\_\_\_\_\_\_\_\_\_\_\_\_\_\_\_\_\_\_\_\_\_\_\_\_\_\_\_\_\_\_\_\_\_\_\_\_\_\_\_\_\_\_\_\_\_\_\_
gru\_44 (GRU)                 (None, None, 200)         240600    
\_\_\_\_\_\_\_\_\_\_\_\_\_\_\_\_\_\_\_\_\_\_\_\_\_\_\_\_\_\_\_\_\_\_\_\_\_\_\_\_\_\_\_\_\_\_\_\_\_\_\_\_\_\_\_\_\_\_\_\_\_\_\_\_\_
batch\_normalization\_83 (Batc (None, None, 200)         800       
\_\_\_\_\_\_\_\_\_\_\_\_\_\_\_\_\_\_\_\_\_\_\_\_\_\_\_\_\_\_\_\_\_\_\_\_\_\_\_\_\_\_\_\_\_\_\_\_\_\_\_\_\_\_\_\_\_\_\_\_\_\_\_\_\_
gru\_45 (GRU)                 (None, None, 200)         240600    
\_\_\_\_\_\_\_\_\_\_\_\_\_\_\_\_\_\_\_\_\_\_\_\_\_\_\_\_\_\_\_\_\_\_\_\_\_\_\_\_\_\_\_\_\_\_\_\_\_\_\_\_\_\_\_\_\_\_\_\_\_\_\_\_\_
batch\_normalization\_84 (Batc (None, None, 200)         800       
\_\_\_\_\_\_\_\_\_\_\_\_\_\_\_\_\_\_\_\_\_\_\_\_\_\_\_\_\_\_\_\_\_\_\_\_\_\_\_\_\_\_\_\_\_\_\_\_\_\_\_\_\_\_\_\_\_\_\_\_\_\_\_\_\_
time\_distributed\_23 (TimeDis (None, None, 29)          5829      
\_\_\_\_\_\_\_\_\_\_\_\_\_\_\_\_\_\_\_\_\_\_\_\_\_\_\_\_\_\_\_\_\_\_\_\_\_\_\_\_\_\_\_\_\_\_\_\_\_\_\_\_\_\_\_\_\_\_\_\_\_\_\_\_\_
softmax (Activation)         (None, None, 29)          0         
=================================================================
Total params: 1,326,629
Trainable params: 1,324,629
Non-trainable params: 2,000
\_\_\_\_\_\_\_\_\_\_\_\_\_\_\_\_\_\_\_\_\_\_\_\_\_\_\_\_\_\_\_\_\_\_\_\_\_\_\_\_\_\_\_\_\_\_\_\_\_\_\_\_\_\_\_\_\_\_\_\_\_\_\_\_\_
None
--------------------------------------------------------------------------------
True transcription:

stuff it into you his belly counselled him
--------------------------------------------------------------------------------
Predicted transcription:

stofitento you his bely cocald im
--------------------------------------------------------------------------------

    \end{Verbatim}

    One standard way to improve the results of the decoder is to incorporate
a language model. We won't pursue this in the notebook, but you are
welcome to do so as an \emph{optional extension}.

If you are interested in creating models that provide improved
transcriptions, you are encouraged to download
\href{http://www.openslr.org/12/}{more data} and train bigger, deeper
models. But beware - the model will likely take a long while to train.
For instance, training this
\href{https://arxiv.org/pdf/1512.02595v1.pdf}{state-of-the-art} model
would take 3-6 weeks on a single GPU!


    % Add a bibliography block to the postdoc
    
    
    
    \end{document}
